%%%% Modèle proposé par frederic.mazaleyrat@ens-paris-saclay.fr %%%%
%%%% 31/01/2017 %%%%

\documentclass[12pt,a4paper]{book}
\usepackage[utf8]{inputenc}
\usepackage[T1]{fontenc}
\usepackage{amsmath}
\usepackage{amsfonts}
\usepackage{fancyhdr}
\usepackage{amssymb}
\usepackage{color} % où xcolor selon l'installation
\definecolor{Prune}{RGB}{99,0,60}
\usepackage{mdframed}
\usepackage{multirow} %% Pour mettre un texte sur plusieurs rangées
\usepackage{multicol} %% Pour mettre un texte sur plusieurs colonnes
\usepackage{scrextend} %Forcer la 4eme  de couverture en page pair
\usepackage{tikz}
\usepackage{graphicx}
\usepackage[absolute]{textpos} 
\usepackage{colortbl}
\usepackage{array}
%\RequirePackage{geometry}% That nicely create a one-page template
%\geometry{textheight=100ex,textwidth=40em,top=30pt,headheight=30pt,headsep=30pt,inner=80pt}
\usepackage{geometry}
\begin{document}

\begin{titlepage}


%\thispagestyle{empty}

\newgeometry{left=7.5cm,bottom=2cm, top=1cm, right=1cm}

\tikz[remember picture,overlay] \node[opacity=1,inner sep=0pt] at (-28mm,-135mm){\includegraphics{Bandeau_UPaS.pdf}};

% fonte sans empattement pour la page de titre
\fontfamily{fvs}\fontseries{m}\selectfont


%*****************************************************
%******** NUMÉRO D'ORDRE DE LA THÈSE À COMPLÉTER *****
%******** POUR LE SECOND DÉPOT                   *****
%*****************************************************

\color{white}

\begin{picture}(0,0)

\put(-150,-735){\rotatebox{90}{NNT: 2020UPASA000}}
\end{picture}
 
%*****************************************************
%**  LOGO  ÉTABLISSEMENT PARTENAIRE SI COTUTELLE
%**  CHANGER L'IMAGE PAR DÉFAUT **
%*****************************************************
\vspace{-10mm} % à ajuster en fonction de la hauteur du logo
\flushright \includegraphics[scale=1]{logo2.png}




%*****************************************************
%******************** TITRE **************************
%*****************************************************
\flushright
\vspace{10mm} % à régler éventuellement
\color{Prune}
\fontfamily{fvs}\fontseries{m}\fontsize{22}{26}\selectfont
  Titre de la thèse (sur plusieurs lignes si nécessaire, 4 voire 5)

%*****************************************************

%\fontfamily{fvs}\fontseries{m}\fontsize{8}{12}\selectfont
\normalsize
\vspace{1.5cm}

\color{black}
\textbf{Thèse de doctorat de l'Université Paris-Saclay}

\vspace{15mm}

École doctorale n$^{\circ}$ 000, dénomination et sigle\\
\small Spécialité de doctorat: voir annexe\\
\footnotesize Unité de recherche: voir annexe\\
\footnotesize Référent: : voir annexe
\vspace{15mm}

\textbf{Thèse présentée et soutenue à ....., le .... 202X, par}\\
\bigskip
\Large {\color{Prune} \textbf{Prénom NOM}}


%************************************
\vspace{\fill} % ALIGNER LE TABLEAU EN BAS DE PAGE
%************************************

\flushleft \small \textbf{Composition du jury:}
\bigskip



\scriptsize
\begin{tabular}{|p{8cm}l}
\arrayrulecolor{Prune}
\textbf{Prénom Nom} &   Président/e\\ 
Titre, Affiliation & \\
\textbf{Prénom Nom} &  Rapportrice \\ 
Titre, Affiliation   &   \\ 
\textbf{Prénom Nom} &  Rapporteur \\ 
Titre, Affiliation  &   \\ 
\textbf{Prénom Nom} &  Examinatrice \\ 
Titre, Affiliation   &   \\ 
\textbf{Prénom Nom} &  Examinateur \\ 
Titre, Affiliation   &   \\ 
\textbf{Prénom Nom} &  Examinateur \\ 
Titre, Affiliationt   &   \\ 

\end{tabular} 

\medskip
\begin{tabular}{|p{8cm}l}\arrayrulecolor{white}
\textbf{Prénom Nom} &   Directrice\\ 
Titre, Affiliation & \\
\textbf{Prénom Nom} &   Codirecteur\\ 
Titre, Affiliation  &   \\ 
\textbf{Prénom Nom} &   Coencadrante\\ 
Titre, Affiliation  &   \\ 
\textbf{Prénom Nom} &  Invité \\ 
Titre, Affiliation  &   \\ 


\end{tabular} 


\end{titlepage}
%%%%%%%%%%%%%%%%%%%%%%%%%%%%%%%%%%%%%%%%%%%%%%%%%%%%%%%%%%%%%%%
% 4eme de couverture
\ifthispageodd{\newpage\thispagestyle{empty}\null\newpage}{}
\thispagestyle{empty}
\newgeometry{top=1.5cm, bottom=1.25cm, left=2cm, right=2cm}
\fontfamily{rm}\selectfont

\lhead{}
\rhead{}
\rfoot{}
\cfoot{}
\lfoot{}

\noindent 
%*****************************************************
%***** LOGO DE L'ED À CHANGER ÉVENTUELLEMENT *********
%*****************************************************
\includegraphics[height=2.45cm]{EOBE}
\vspace{1cm}
%*****************************************************

\begin{mdframed}[linecolor=Prune,linewidth=1]
\vspace{-.25cm}
\paragraph*{Titre:} Contributions à l'étude et la réalisation de composants magnétiques monolithiques réalisés par PECS/SPS et à leurs applications en électronique de puissance

\begin{small}
\vspace{-.25cm}
\paragraph*{Mots clés:} Transformateur, Coupleur, Ferrite, Frittage PECS/SPS, VRM

\vspace{-.5cm}
\begin{multicols}{2}
\paragraph*{Résumé:} Dans le cadre de l'intégration de puissance, il est nécessaire de réduire la taille des composants magnétiques. Deux stratégies peuvent alors être adoptées : la première consiste à augmenter la fréquence de travail et donc à développer des matériaux fonctionnant à plus haute fréquence, la seconde consiste à modifier la conception des composants en changeant la répartition des enroulements par rapport au circuit magnétique. Sur la base de ces deux stratégies, des ferrites sont optimisés via leur composition chimique et fabriqués grâce au procédé de frittage PECS/SPS (Pulsed Electric Current Sintering / Spark Plasma Sintering). Ce procédé permet de diminuer suffisamment la température de frittage pour pouvoir placer du cuivre massif dans le ferrite pendant le frittage. Ceci a permis la réalisation de transformateurs et de coupleurs monolithiques qui sont des composants idéaux pour l'intégration de puissance car le volume qu'ils occupent est entièrement utilisé. Enfin, un prototype d'alimentation VRM 2~MHz utilisant ces composants est réalisé, permettant d'atteindre une densité de puissance de 15~kW/l pour une puissance de sortie de 200~W.
\end{multicols}
\end{small}
\end{mdframed}

\begin{mdframed}[linecolor=Prune,linewidth=1]
\vspace{-.25cm}
\paragraph*{Title:} Contributions to the study and realization of monolithic magnetic components made by PECS / SPS and their applications in power electronics

\begin{small}
\vspace{-.25cm}
\paragraph*{Keywords:} Transformer, Coupler, Ferrite, PECS/SPS Sintering, VRM

\vspace{-.5cm}
\begin{multicols}{2}
\paragraph*{Abstract:} In the context of power integration, it is necessary to reduce the size of the magnetic components. Two strategies can be adopted: the first is to increase the switching frequency and thus to develop materials operating at higher frequency; the second consists in modifying the design of the components by changing the winding distribution with respect to the magnetic circuit. With these two strategies, ferrites are optimized by their chemical composition and manufactured using the PECS/SPS (Pulsed Electric Current Sintering / Spark Plasma Sintering) sintering process. This method allows to reduce the sintering temperature sufficiently to be able to place solid copper in the ferrite during sintering. This also allowed the realization of monolithic transformers and couplers which are ideal components for power integration because the volume they occupy is fully used. Finally, a 2~MHz VRM power supply prototype using these components is produced, achieving a power density of 15~kW/l with an output power of 200~W.
\end{multicols}
\end{small}
\end{mdframed}

%************************************
\vspace{3cm} % ALIGNER EN BAS DE PAGE
%************************************
\fontfamily{fvs}\fontseries{m}\selectfont
\begin{tabular}{p{14cm}r}
\multirow{3}{16cm}[+0mm]{{\color{Prune} Université Paris-Saclay\\
Espace Technologique / Immeuble Discovery\\
Route de l’Orme aux Merisiers RD 128 / 91190 Saint-Aubin, France}} & \multirow{3}{2.19cm}[+9mm]{\includegraphics[height=2.19cm]{e.pdf}}\\
\end{tabular}

\end{document}