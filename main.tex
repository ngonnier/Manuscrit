%Documentclass thesul modifée pour la page de garde : thesul-cs
\documentclass[11pt]{thesul-cs}
\usepackage[utf8]{inputenc}
\usepackage[T1]{fontenc}
\usepackage{amsmath}
\usepackage{amsfonts}
\usepackage{arydshln}
\usepackage{booktabs}
%\usepackage{tablefootnote}
\usepackage{threeparttable}
%\pagestyle{ThesisHeadings}
\pagestyle{ThesisHeadings}
\setlength{\HeadRuleWidth}{0.4pt}
% %Fancy headers
% % \usepackage{fancyhdr}
% %\pagestyle{Fancy}
% %fancyhf{}
% \OddHead[LE,RO]{{\thepage}{}}
% \EvenHead[LE,RO]{\thepage}
% % \fancyhead[LO]{\rightmark}
% % \fancyhead[RE]{\leftmark}

%mini table of contents
\usepackage[french]{minitoc}
\usepackage{amssymb}
\usepackage{xcolor} % où xcolor selon l'installation
\usepackage{mdframed}
\usepackage{multirow} %% Pour mettre un texte sur plusieurs rangées
\usepackage{multicol} %% Pour mettre un texte sur plusieurs colonnes
\usepackage{scrextend} %Forcer la 4eme  de couverture en page pair
\usepackage{tikz}
\usepackage[absolute]{textpos}
\usepackage{colortbl}
\usepackage{amsmath}
\usepackage{amsthm}
\usepackage{stmaryrd}
\usepackage{array}
\usepackage{url}
\usepackage[ruled,french,frenchkw,onelanguage]{algorithm2e}
\usepackage{mathtools}
\usepackage{subfiles}
\usepackage{float}
\usepackage{setspace}
\usepackage{refcount}% required for \getpagerefnumber
\usepackage{xr}% required to reference labels in external documents

\SetRealMargins{30mm}{20mm}
\renewcommand{\floatpagefraction}{.8}
\renewcommand{\baselinestretch}{1.25}
\DefineBibliographyStrings{french}{
  %  backrefpage={see p.},
  backrefpage={},
  %  backrefpages={see pp.}
  backrefpages={}
}

\addbibresource{./Biblio/biblio.bib}


%---------------------------
% (1) Weights
%---------------------------
\newcommand\w{\omega}

%---------------------------
% (2) External inputs
%---------------------------
\newcommand\inpx{X}

%---------------------------
% (3) Contextual inputs
%---------------------------

\newcommand\inpc{\gamma}

%---------------------------
% (4) Contextual suffix
%---------------------------
\newcommand\cont{_{c}}

%---------------------------
% (5) External suffix
%---------------------------

\newcommand\ext{_{e}}

%---------------------------
% (6) Map index
%---------------------------
\newcommand\m[1]{^{({#1})}}

%---------------------------
% (7) BMU
%---------------------------

\newcommand\bmu{\Pi}

%---------------------------
% (7) Neighborhood radius
%---------------------------
\newcommand\h{h}

\DeclareMathOperator*{\argmax}{arg\,max}

%compile only a chapter
\SetAlCapSkip{1em}
\SetKwInput{KwInput}{Input}
\SetKwInput{KwOutput}{Output}

\DeclareMathOperator{\sign}{sgn}
\newtheorem{propriete}{Propriété}
\newtheorem{proposition}{Proposition}
\parskip=5pt

\begin{document}
\dominitoc
\ThesisTitle{Auto-organisation Décentralisée Multi-Cartes}
\ThesisDate{ ? décembre 2022}
\ThesisAuthor{Noémie Gonnier}
\ThesisUL
% Jury:
\President = {Le président &du jury}
\Rapporteurs = {Le rapporteur 1 &du laboratoire\\
Le rapporteur 2\\
Le rapporteur 3}
\Examinateurs = {L’examinateur 1\\
L’examinateur 2}
\MakeThesisTitlePage

\begin{ThesisAbstract}
  \begin{FrenchAbstract}
Le cortex cérébral apparaît dans de nombreux travaux comme une architecture de modules autonomes, les aires corticales, connectées rétroactivement. Elles échangent des informations sensorielles ou de plus haut niveau et implémentent des tâches d'apprentissage extrêmement sophistiquées. Cette notion bio-inspirée d'architecture modulaire présente un intérêt computationnel dans la recherche de nouveaux paradigmes d'apprentissage. Il s'agit en effet de systèmes complexes, propices à faire émerger des mécanismes d'apprentissage dus à l'interaction entre les modules.
Partant de cette inspiration biologique, cette thèse propose d'étudier la création d'architecture modulaire non hiérarchique de cartes auto-organisatrices.
Les cartes auto-organisatrices sont un algorithme d'apprentissage non-supervisé permettant de représenter de façon ordonnée et en faible dimension un espace d'entrées quelconques. Cet algorithme s'inspire de l'organisation présente dans les aires corticales. 
Par la simplicité de leurs règles de mise à jour et leur capacité de représenter chaque entrée par une position, les cartes nous apparaissent comme des candidates naturelles à la conception d'une architecture modulaire.
Nous développons et étudions dans ces travaux un modèle modifié de cartes auto-organisatrices permettant de les associer au sein d'une architecture non hiérarchique. Nous appelons ce modèle CxSOM, pour \emph{Consensus Driven Multi-SOM}.
Cette thèse constitue ensuite une analyse expérimentale des mécanismes d'organisation et d'apprentissage émergeant de l'association des modules. Nous nous concentrons sur la mise en évidence de mécanismes de mémoire associative entre modalités~; l'objectif est de pouvoir apprendre une représentation de plusieurs espaces d'entrées au sein de l'architecture et d'extraire des relations existant entre ces entrées.
Pour analyser ces mécanismes, nous mettons l'accent sur une méthode de représentation des réponses de l'architecture, et proposons des outils de visualisation et de mesure de l'apprentissage. Grâce à ce cadre expérimental, nous avons pu mettre en lumière des comportements d'apprentissage associatifs spécifiques à ces architectures et des perspectives d'étude possibles.
En particulier, le modèle présente un comportement de prédiction d'entrée, rendu possible par les interactions entre les modules de l'architecture.
La proposition du modèle CxSOM et l'analyse des comportements sur des architectures simples nous permettent d'élaborer une base de travail, vers la conception d'architectures non hiérarchiques comportant de nombreuses cartes.

  \KeyWords{Cartes auto-organisatrices, architecture modulaire, mécanismes d'apprentissage, bio-inspiration}
  \end{FrenchAbstract}

  \begin{EnglishAbstract}
The cerebral cortex appears in many models as an architecture of autonomous modules connected retroactively, exchanging sensory and higher-level information and implementing extremely sophisticated learning tasks. 
This bio-inspired concept of a modular architecture has also a computational interest in the search for new learning paradigms, especially for the design of autonomous learning networks. They are indeed complex systems, which can conduct to the emergence of learning mechanisms due to the interaction between the modules.
This thesis creates a non-hierarchical architecture model of self-organizing maps.
Self-organizing maps are an unsupervised learning algorithm that creates an ordered representation of any input space onto a low-dimensional space, and is inspired by the organization present in cortical areas. 
Due to the simplicity of their update rules and their ability to map an input to a position, they appear to us as natural candidates for the design of a modular architecture.
In this work, we introduce a modified self-organizing map model and an interface method for associating them within a non-hierarchical architecture. We call this model CxSOM, standing for Consensus-Driven Multi-SOM.
The thesis constitutes an experimental analysis of the organization and learning mechanisms emerging from the association of those modules. 
We focus on highlighting associative memory mechanisms between several modalities; the goal for the model is to learn a representation of multiple input spaces within the architecture, as well as the relationships existing between these inputs.
To analyze these mechanisms, we focus on a method to represent the architecture's responses, and propose visualization and learning measurement tools. Through this analysis framework, we were able to highlight specific learning behaviors of the architecture model. Particulary, we show an input prediction behavior that emerges from the interactions between the maps of the architecture.
The construction of the CxSOM model and the analysis of its behavior on simple architectures stand as groundwork towards the design of a non-hierarchical architecture model with numerous maps.
   
    \KeyWords{Self-organizing maps, modular architecture, learning mecanisms, bio-inspiration}
    \end{EnglishAbstract}
  \end{ThesisAbstract}

  \begin{ThesisAcknowledgments}
    
 Je tiens tout naturellement à remercier mes directeurs de thèse, qui proposé ce sujet vaste et accompagnée dans ce travail de recherche. Je remercie également les membres du jury qui ont consacré du temps pour la lecture et les retours de ce manuscrit.


S'il est des gens à remercier tout particulièrement pour l'existence de cette thèse, ce sont bien
les collègues du LORIA. Dans un contexte de covid menant la majorité de ces travaux à avoir été réalisés seule derrière mon PC, les discussions et échanges avec les autres membres du laboratoire ont été salvateurs.
Je remercie Adrien, Pierre et Yann pour toutes les discussions scientifiques, politiques, philosophiques et légères qui ont enrichi ces années dans ce bureau et ailleurs. Je remercie également Athénaïs qui a toujours cru en moi et dont la passion débordante et communicative pour la science m'a toujours motivée. Merci enfin également à toustes les membres du laboratoire que j'ai pu rencontrer pendant ces quatre ans et qui ont permis à cette thèse de ne pas être un pénible travail solitaire et à qui je souhaite toute la réussite pour la suite.
Merci enfin à tous ceux et celles qui ont partagé ma vie à Nancy, \'Edith, Solenn, Anaïs, Hugo, Sophie ... pour m'avoir sorti la tête des lignes de code.

\end{ThesisAcknowledgments}

\setcounter{tocdepth}{1}
\tableofcontents
\subfile{00-Introduction/source}
\mainmatter
\subfile{02-Architectures/source}
\subfile{03-SOM/source}
\subfile{07-Relaxation/source}
\subfile{04-Representation/source}
\subfile{06-Analyse/source}
\subfile{05-Indicateur/source}
\subfile{06b-Analyse-2D/source}
\subfile{Conclusion}
%Bibliography files

\nocite{gonnier2020}
\nocite{Gonnier2021InputPU}

\printbibliography[keyword=mypublis, title={Publications}]
\section*{Code}
Le code source des expériences présentées dans ce manuscrit est rassemblé en~\url{https://github.com/ngonnier/cxsom-expes}

\printbibliography[title={Bibliographie}]
\end{document}