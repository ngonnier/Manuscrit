\chapter*{Introduction générale}

Les systèmes biologiques qui nous entourent présentent une incroyable diversité de structures leur permettant d'évoluer et de s'adapter à leur environnement.
Des systèmes en apparence simples présentent ainsi des capacités d'optimisation de tâches. Le blob est par exemple un être unicellulaire capable de s'étendre sur plusieurs mètres. Uniquement grâce aux communications chimiques opérant au sein de la membrane, il est capable de trouver le chemin le plus court dans un labyrinthe entre deux points sur lesquels sont placés de la nourriture \cite{Nakagaki2000IntelligenceMB}, de retrouver une configuration optimale d'un réseau de transport et sont même capables d'apprentissage~: deux entités fusionnant se transmettent des connaissances de leur environnement, montrant qu'elles ont appris à un point de l'information sur ce dernier.
Les colonies de fourmis, quant à elles constituées de milliers, voire de millions d'individus, sont capables de s'auto-organiser pour effectuer des tâches complexes de coopération pour construire leur nid, se défendre face aux prédateurs et trouver leur nourriture via une communication par leurs phéromones, son et toucher \cite{jackson_communication_2006}.
Autrement dit, ces systèmes biologiques présentent des capacités de calcul remarquables. 
Toutes ces stratégies mises en place par des systèmes biologiques ont inspiré de nombreux algorithmes d'optimisation imitant les colonies de fourmis, les essaims d'abeilles, les groupes de chats, les bancs de poissons ou encore les baleines, tous ces groupes d'animaux présentant des méthodes de communication décentralisées efficaces pour accomplir une tâche donnée~\cite{Darwish2018BioinspiredCA}.

Un parangon de système biologique de calcul est sans conteste le cerveau humain ou animal, qui est capable d'exécuter des tâches de calcul et d'apprentissage incroyablement sophistiquées via une multitude de signaux électriques et chimiques circulant entre les neurones et au sein des vaisseaux sanguins.
L'inspiration biologique occupe ainsi une place de premier rang dans les débuts de la recherche en intelligence artificielle.
Face à des modèles symboliques, les modèles à base d'apprentissage ont été initialement développés en s'appuyant sur le modèle biologique du neurone, afin de chercher à imiter les capacités d'évolution et d'adaptation à l'origine de la notion d'apprentissage dans les réseaux de neurones biologiques.
Le perceptron, modèle à l'origine des réseaux de Deep Learning les plus performants à l'heure actuelle, s'inspirait par exemple du modèle de neurone biologique~\cite{McCulloch1990ALC}.
La diversité des applications actuelles de l'apprentissage automatique et le développement de nombreux modèles très performants ont amené la recherche en apprentissage automatique à se concentrer principalement sur une approche algorithmique et computationnelle de ces réseaux, en particulier pour les architectures de Deep Learning. 
Cet éloignement de la biologie permet de s'extraire des contraintes physiques liées au neurone pour envisager des règles de calcul adaptées à une tâche spécifique.
Néanmoins, la biologie, par sa diversité de comportements encore incompris, reste une source d'inspiration abondante pour apporter des paradigmes alternatifs ou complémentaires aux modèles d'apprentissage existants dans l'état de l'art. Le comportement du cerveau est loin d'être entièrement compris et modélisé, ce qui signifie que les possibilités d'inspiration biologique sont constamment en train d'évoluer.
Les réseaux de neurones impulsionnels (\emph{Spiking Neural Networks}) sont un exemple de modèle d'apprentissage illustrant une complémentarité récente entre l'approche bio-inspirée et l'approche computationnelle.
Ces réseaux de neurones ont été développés dès les années 1990~\cite{Maass1996NetworksOS} et s'appuient directement sur le modèle biologique de neurone.
Ils apparaissent pourtant dans de nombreux travaux récents comme une méthode montante dans le domaine de l'apprentissage automatique pour la conception de modèles d'apprentissage moins énergivores et distribués, grâce à la conception d'architectures matérielles neuromorphiques telles que LOIHI \footnote{\url{https://www.intel.com/content/www/us/en/research/neuromorphic-computing.html}}. De nombreux travaux cherchent ainsi à adapter des réseaux de neurones classiques de l'état de l'art dans une version impulsionnelle, faisant ainsi passer les SNN de la biologie au calcul~\cite{Schuman2022OpportunitiesFN}.
Sans chercher à concurrencer l'état de l'art, de tels modèles apportent de nouveaux paradigmes de calcul pouvant se combiner avec des approches plus appliquées.
Nous pensons ainsi qu'il est pertinent de continuer à explorer des modèles d'apprentissage automatique inspirés de la biologie.


La thèse que nous présentons présente un modèle d'apprentissage élaboré à la fois dans une démarche d'inspiration biologique et utilisant des techniques plutôt calculatoires.
Pour l'aspect biologique, de nombreuses modélisations générales du cortex cérébral, telle que~\cite{binzegger05, Meunier2009HierarchicalMI,sporns_structure_2013,betzel_multi-scale_2017} proposent que le cortex est une architecture composée de modules auto-organisés.
Ces modules communiquent autour des informations sensorielles collectées par l'organisme. Cette communication est réalisée de façon interne, liant des informations temporelles, sensorielles et abstraites provenant de différentes parties du cortex et à différentes échelles spatiotemporelles. Enfin, bien qu'une hiérarchie de traitement de l'information apparaît entre ces modules, certains traitant des entrées sensorielles et d'autres des entrées plus abstraites, de nombreux circuits de rétroactions entre les modules semblent présents à différents niveaux de l'architecture.
La recherche d'architectures cognitives s'inspirant de l'architecture du cortex cérébral est un enjeu de longue date dans la recherche en apprentissage automatique. Il s'agit de développer des réseaux de neurones autonomes, capables de mémoire et de prise de décision de façon non supervisée, qui s'inspirent des architectures modulaires présentes dans le cerveau humain et qui cherchent à imiter certains comportements~; un historique de ces modèles est retrouvable en~\cite{Kotseruba201840YO}.
Le développement de la robotique appelle également à envisager des modèles d'apprentissage directement liés à la perception sensorielle, en s'inspirant des architectures modulaires existant dans le cortex et qui permettent le traitement de cette perception.

Dans cette inspiration d'architecture inspirée de l'architecture corticale, nous proposons dans cette thèse un modèle d'architecture d'apprentissage modulaire non-hiérarchique.
On entend ici par système modulaire un système composé d'une multiplicité de sous-structures interchangeables qui interagissent entre elles par le biais d'interfaces définies. 
Dans une architecture modulaire, ces sous-structures communiquent de façon locale, sans être supervisées par un processus externe. Cette propriété de modularité est partagée par de nombreux systèmes biologiques et artificiels et présente des avantages en termes de réutilisation, de robustesse aux fautes, de redondance et de traitement local de l'information \cite{clune_evolutionary_2013}.


Pour construire cette architecture modulaire, nous nous appuierons sur un modèle d'apprentissage existant inspiré de la biologie~: les cartes auto-organisatrices (SOM) \cite{Kohonen1982}.
La SOM est un algorithme de quantification vectorielle.
L'apprentissage d'une SOM sur des données consiste à mettre à jour ces poids afin de représenter la disposition des données d'entrée sur les poids de cette grille 2D.
Pour cela, les règles d'évolution reposent sur un principe de compétition \emph{Winner Take All}, permettant de choisir un poids, le \emph{Best Matching Unit} représentant le mieux une entrée. La mise à jour est ensuite effectuée en collaboration entre neurones~: le gagnant est mis à jour ainsi que ses voisins proches. Cette dualité compétition/collaboration entraîne une organisation finale de la carte sur l'espace d'entrée ordonnée~: deux n\oe{}uds proches ont des poids proches. L'organisation conserve également d'autres propriétés topologiques, et est finalement une représentation en 2D de l'espace d'entrée.
Les cartes de Kohonen sont initialement inspirées des cartes topographiques présentes dans les cortex sensoriels ou moteurs. Les mécanismes de choix du BMU et de mise à jour, d'abord calculés grâce à des champs neuronaux, ont ensuite évolué vers des règles plus computationnelles de calcul de distance et d'argmin sur la carte.
De nombreux travaux ont cherché à associer les SOMs en architecture, que nous passerons en revue dans l'état de l'art de cette thèse, mais peu ont exploré l'aspect topologique et la simplicité des règles de calcul d'une carte pour les assembler en architectures modulaires comportant des rétroactions.


Outre l'aspect d'inspiration biologique, la construction de systèmes d'apprentissage modulaires a également une motivation computationnelle. 
Elle découle de l'observation que la combinaison de principes et d'algorithmes simples en architecture modulaire peut mener au développement de mécanismes de calcul complexes au sein du système.
Les structures modulaires ont en effet la particularité d'engendrer des comportements émergents dans leur dynamique, c'est-à-dire que le comportement global du système résulte bien de l'interaction entre les modules et non seulement de la somme des comportements des modules pris individuellement~: il s'agit de systèmes complexes.
Dans le cas de systèmes modulaires, le comportement général qui émerge est une organisation des modules de l'architecture~: on parle d'auto-organisation du système. Il peut s'agir d'une synchronisation, d'une collaboration ...
Cette auto-organisation est effectuée de façon décentralisée, via des règles et interactions locales au sein de l'architecture modulaire~: elle n'est pas supervisée par un processus extérieur à l'architecture.
Des exemples célèbres d'émergence de comportement dans des systèmes artificiels sont le jeu de la vie, ou le deep learning, et l'intelligence d'essaim.
Dans cette thèse, nous nous plaçons ainsi dans une démarche de construction d'un système d'apprentissage~: l'assemblage de module ayant des règles simples d'évolution peut mener à des comportements collectifs plus complexes. 
Ici, les comportement complexes auxquels nous nous attendons sont des comportements d'apprentissage. Nous savons que chaque module, une carte de Kohonen, est capable d'apprendre une représentation de son espace d'entrée. Nous nous attendons à diversifier les comportements d'apprentissage.

Ainsi, la conception d'une architecture modulaire d'apprentissage s'inspire ainsi d'une part des propriétés de modularité et d'émergence observées dans différents systèmes naturels, en particulier le cerveau, et artificiels comme l'intelligence d'essaim, les réseaux de neurones profonds ou le jeu de la vie.
La propriété de modularité permettrait de réaliser des calculs complexes en exploitant la collaboration des modules du système et en générant une auto-organisation au sein du système. 
Dans cette démarche, les SOMs sont de bonnes candidates pour être associés en architectures modulaires. Elles sont en effet inspirées de l'organisation des aires corticales, aires qui sont associées dans le cerveau en architecture non-hiérarchique. 
L'introduction de rétroactions et de connexions temporelles pourrait par ailleurs permettre d'explorer de nouveaux comportements computationnels.

\section*{Problématique de la thèse}

Enfin, le modèle d'apprentissage que nous présentons dans cette thèse se base sur les travaux précédents réalisés dans notre équipe de recherche. Ces travaux ont associé des champs neuronaux dynamiques couplés entre eux à des SOMs afin d'ajouter un aspect d'apprentissage à ces réseaux dynamiques.
Les DNFs sont très proches des SOMs par leur notion de voisinage dans le calcul de l'activation et le mécanisme de Winner Take All en résultant, remplacé par un argmax dans une carte de Kohonen. Le couplage rétroactif entre les SOMs dans ces travaux engendre une réponse dynamique collective des DNFs qui évoluent pour se stabiliser vers une position de consensus pour définir les BMUs des cartes.
Dans cette thèse, nous utilisons un mécanisme de relaxation inspiré des champs neuronaux dynamiques, cherchant les BMUs à une position de consensus dans l'architecture.
En particulier, nous exploiterons l'aspect topologiquement ordonné d'une carte en utilisant la position du BMU comme information transmise entre cartes. 

Enfin, la notion d'architecture cognitive présentée plus haut cherche à explorer de grands principes liés à la cognition tels que l'apprentissage autonome, sans supervision, le traitement de données temporelles, l'apprentissage sur le long terme sans oubli catastrophique des données précédentes et la fusion de données multimodales, s'inspirant du traitement multisensoriel du cerveau humain.
L'objectif de nos travaux est de proposer un modèle de carte qui puisse être utilisée en tant que module, de définir l'interface entre les modules afin de créer une architecture et de comprendre les comportements de calcul qui émergent de l'association des modules. 
Nous nous avons choisi de nous concentrer en particulier sur l'apprentissage associatif de données multimodales. Il s'agit d'apprendre les relations existant entre les entrées, en plaçant cet apprentissage de relations à un niveau interne à l'architecture.
Cette mémoire associative apparaît comme un type d'apprentissage non supervisé, et le principe est d'apprendre une représentation des entrées mais également de leurs relations.



Le but de la thèse est ainsi d
\begin{itemize}
    \item Quelles interfaces entre SOMs sont pertinentes à mettre en place pour la construction d'une architecture modulaire ?
    \item En ayant défini un modèle d'architecture modulaire, quels sont les comportements en résultant et comment les utiliser dans un contexte de mémoire associative ?
\end{itemize}

Ces méthodes interdisciplinaire, entre inspiration biologique et approche computationnelle, nous ont amenés à proposer un modèle d'architecture non-hiérarchique de cartes de Kohonen, CxSOM, que nous présentons et détaillons dans cette thèse.


\section*{Contributions et plan}

Cette thèse cherche donc, dans un cadre bio-inspiré, à construire une architecture modulaire décentralisée de cartes auto-organisatrices. L'idée de cette approche est de rechercher des nouveaux comportements d'apprentissage émergeant de l'interaction entre les modules d'une grande architecture, à l'inverse des méthodes plus ingénieures consistant à diviser une tâche connue en sous-systèmes.
Le manuscrit est organisé de la façon suivante.
Le chapitre~\ref{chap:archis} présente un état de l'art des architectures de cartes auto-organisatrices existantes afin de définir ce qu'on entend par architecture modulaire décentralisée et positionner notre modèle dans l'ensemble des modèles existants. Nous nous intéresserons en particulier aux interfaces choisies entre les modules.

Nous détaillerons ensuite le modèle d'architecture décentralisée de cartes auto-organisatrices que nous proposons dans cette thèse, CxSOM (\emph{Consensus-driven Multi-SOM}) au chapitre~\ref{chap:modele}.
Ce modèle d'architecture permet d'associer des cartes en architecture non-hiérarchiques. Dans ce modèle, les activités des cartes sont interdépendantes et l'apprentissage s'appuie sur une recherche de consensus entre les cartes pour la recherche d'un BMU, par un processus dynamique inspiré de la relaxation entre DNF.
Si le modèle a pour but à long terme de concevoir une architecture comportant de nombreux modules ainsi que des connexions temporelles, nous avons concentré cette thèse sur l'analyse des comportements d'architecture de deux et trois cartes.
Le but de cette thèse est de proposer une méthodologie d'analyse de ce modèle et d'en tirer des comportements élémentaires qui serviront à poser les jalons de la construction d'une architecture plus grande.
Les expériences présentées dans la suite du manuscrit étudient le comportement de ces architectures de différents angles d'approche.
Le chapitre~\ref{chap:relaxation} présente une analyse plus approfondie du mécanisme de relaxation entre les BMUs, posé comme l'interface entre carte. 
Ce chapitre cherche à répondre à la pertinence de ce mécanisme en tant qu'algorithme de choix de BMU pour l'apprentissage.

Nous proposerons ensuite une méthode expérimentale et des représentations rapprochant l'architecture de cartes de modèles d'apprentissage communs au chapitre \ref{chap:repr}.
Nous présenterons ensuite les comportements élémentaires observés sur des architectures de deux et trois cartes en une dimension, qui sont plus facile à visualiser. 
Nous présenterons notamment un comportement de prédiction rendu possible par le modèle.
Nous proposons au chapitre~\ref{chap:indicateur} des indicateurs numériques originaux d'évaluation de l'apprentissage associatif par l'architecture de cartes, dans le but d'étendre l'analyse du modèle à des architectures difficilement représentables visuellement.
Le chapitre \ref{chap:analyse2D} utilise la méthodologie pour analyser le comportement d'architectures de cartes en deux dimensions afin de saisir la scalabilité du modèle.












% Le terme calcul désigne les procédés abstraits utilisés pour le traitement de l'information. 
% Un système réalisant de tels procédés est alors désigné comme système de calcul. 
% D'un point de vue abstrait, ces systèmes peuvent ainsi être biologiques comme artificiels.
% Aussi de nombreux systèmes biologiques excellent à faire du calcul dès lors qu'ils interagissent avec leur environnement et s'adaptent à ce dernier~: 
% les réseaux mycellaires communiquent des informations au sein du réseau et échangent avec leur environnement~; les essaims d'abeilles sont capables de cartographier leur environnement grâce aux informations échangées entre individus. 
% Le blob est un être unicellulaire capable de s'étendre, qui est capable de résoudre des problèmes de recherche de plus court chemin, et ce seulement à partir de signaux chimiques circulant au sein de la cellule.
% Enfin, le cerveau est bien entendu un système de calcul~: il transforme une multitude de signaux physiques et chimiques provenant des capteurs du corps humain en activité électrique, dont émerge un apprentissage et des actions sur son environnement.
% Cette capacité d'apprentissage et surtout la recherche de son imitation est à l'origine du développement de l'intelligence artificielle et en particulier de l'apprentissage automatique, désignant les systèmes capable d'apprendre et de généraliser des informations à partir des données qui leur sont présentées.
% Les modèles d'apprentissage de Deep Learning les plus performants à l'heure actuelle s'appuient sur le perceptron, modèle inspiré à l'origine des neurones biologiques.
% La diversité des applications actuelles de l'apprentissage automatique et les méthodes de développement de ces algorithmes ont amené la recherche actuelle à se concentrer principalement sur les problèmes algorithmiques et computationnels que sur le développement de réseaux d'inspiration biologique.
% Cependant, la biologie, par sa diversité de comportements encore incompris, peut rester une inspiration dans le domaine des réseaux de neurones et apporte des paradigmes alternatifs pour la conception de structures d'apprentissage.
% Un exemple récent montrant le lien entre l'approche computationnelle et l'approche bio-inspirée se trouve dans les réseaux de neurones impulsionnels ou\emph{Spiking Neural Networks}. Ces réseaux sont directement inspirés du modèle biologique du neurone, et leur développement remonte à la fin des années 1980~\cite{Maass1996NetworksOS}.
% Ils apparaissent pourtant dans de nombreux travaux récents comme une méthode montante dans le domaine de l'apprentissage automatique pour la conception de modèles d'apprentissage moins énergivores et distribués. De nombreux travaux sur ces modèles de réseaux de neurones cherchent maintenant à adapter des réseaux de neurones classiques dans une version impulsionnelle. Il s'agit ici d'un exemple dans lequel l'inspiration biologique donne des alternatives possibles aux réseaux de neurones actuels et associent de façon complémentaire une approche bio-inspirée une approche plus computationnelle.
% D'un autre côté, les mécanismes à l'\oe{}uvre dans le cerveau restent loin d'être compris et modélisés.
% La question des calculs à l'\oe{}uvre dans les systèmes biologiques reste ainsi une thématique de recherche actuelle. 
% En ce sens, les découvertes de ces domaines sont des sources d'inspiration biologiques pertinentes.
% L'inspiration biologique cherche également à pl
% Nous plaçons cette thèse dans cette dynamique de recherche de réseaux de neurones inspirés de la biologie. Le but de ce domaine est d'étudier des mécanismes de calculs alternatifs qui pourront être intégrés et combinés à des approches plus classiques de conception de systèmes d'apprentissage, ou s'appliquer sur des applications robotiques.
% Une propriété globale des systèmes biologique tient dans la modularité et la non-hiérarchie des systèmes.
% On entend, par modularité d'un système sa composition en sous-systèmes autonomes effectuant des tâches différentes et collaborant entre eux. 
% Les systèmes modulaires présentent des avantages de réutilisation, de robustesse aux fautes, de redondance et de traitement local de l'information. Cette construction se retrouve dans les systèmes biologiques, \cite{clune_evolutionary_2013} proposant que ces avantages apportés par la modularité ont favorisé cette propriété lors de l'évolution.
% Cette de modularité est très générale. Nous pouvons en définir plusieurs aspects. 
% Des systèmes modulaires peuvent être composés de composants de structure différente interagissant entre eux. Un ordinateur est par exemple composé d'une multitude de modules spécifiques, conçus pour des tâches séparées et de structures complètement différentes.
% Un système modulaire peut également être composé de modules de même structure interagissant entre eux par des interfaces. 
% Par exemple, les colonies de fourmis sont composées d'individus de même structure.
% Les modules ont une fonction spécifique, qui peut être fournie a priori (fourmis guerrières/ouvrières) ou apprises par le module au cours de son interaction avec l'environnement.
% Si on s'intéresse au cerveau, ce dernier possède une structure a priori, mais le rôle de certaines aires cérébrales peut être redéfini et réappris au cours du temps~: l'aire visuelle d'un individu aveugle est réorganisée pour effectuer le traitement des autres sens de l'individu. 
% La modularité apporte alors une flexibilité au système.

% \subsection*{L'auto-organisation, une propriété bio-inspirée}

% Dans certains systèmes modulaires, le comportement dynamique global du système est souvent plus complexe que le comportement de chaque individu. 
% Lorsque c'est le cas, on parle de système complexe et le comportement global un comportement \emph{émergent} du système modulaire. Cette notion d'émergence, dont nous avons donné des exemples biologiques, a inspiré de nombreux modèles de calcul en intelligence artificielle et robotique.
% La fascination pour les comportements émergeant des systèmes complexes vient du fait que ce comportement global n'est analytiquement pas prévisible à partir des données des comportements élémentaires.
% Un exemple computationnel est l'étude des automates cellulaires dans le jeu de la vie. 
% Malgré des règles élémentaires simples, les automates présentes des capacités de mémoire et de transmission de donnée.
% L'intelligence d'essaim (swarm intelligence) est également une application directe du concept d'émergence à des sytèmes multi-agents qui interagissent localement avec leur environnement. 

% Ici encore, la cognition apparaît comme un comportement émergent de l'activité des neurones. Chaque neurone est construit sur le même modèle et les règles d'évolution de chaque neurone sont similaires. 
% Dans son ensemble, le cerveau présente des comportements de calculs qui lui sont propres.
% Dans ce même cerveau, à  une plus grande échelle, on peut séparer des zones fonctionnelles distinctes dans le cerveau. Chaque zone est de construction similaire, mais effectue une fonction différente.
% Le cerveau n'est pas le seul système biologique présentant ce type de fonctionnemment. Ainsi, les systèmes métaboliques ou d'expression de gènes sont des exemples de systèmes aggrégeant des éléments.
% Les colonies de fourmi sont constituées de milliers d'individus effectuants des actions à leur échelle et communiquant localement. Le comportement de la colonie est un système de calcul, capable de résoudre des problème d'optimisation de chemin vers une source de nourriture.

% Les réseaux biologiques hautement modulaires présentant cette capacité de traitement de l'information présentent également des propriétés d'auto-organisation; \cite{Siebert2020RoleOM} suggère même que la modularité est un élément clé pour la présence de motifs auto-organisés. 

% L'apprentissage est un exemple de comportement émergent d'un système. Perceptron multicouches comme comportement émergent de plusieurs couches.
% D'un point de vue modulaire, un exemple de système  = architectures de champs neuronaux dynamiques.
% Apparition de comportements complexes \cite{Sandamirskaya2014DynamicNF}


% \subsection*{Les cartes auto-organisatrices, un module bio-inspiré}

% Nous avons choisi dans cette thèse de s'intéresser à cette deuxième approche~: développer un système modulaire apprenant en associant des réseaux existants connus. 
% Nous nous intéressons spécialement aux cartes auto-organisatrices.

% Si nous revenons à un aspect biologique, les cartes auto-organisatrices sont, par leur comportement un modèle simplifié des aires cérébrales. Les travaux conduits dans notre équipe ces dernières années se sont attachés à construire des architectures modulaires complètement cellulaires. Nous cherchons dans cette thèse à s'inspirer de ces travaux mais en les passant à une échelle moins cellulaire, dans un cadre de simplification du modèle. Cette simplification nous permet une recherche plus facile et moins coûteuse de nouveaux comportements d'apprentissage, tout en restant déclinable si besoin en version cellulaire.
% Dans un cadre d'architecture modulaire,

% Les cartes auto-organisatrices et notamment le modèle de Kohonen sont largement utilisées en tant qu'algorithme d'apprentissage non supervisé appliqué à des tâches de réduction de dimension, de visualisation de données ou de classification.
% De nombreux travaux étudient l'utilisation de plusieurs cartes collaborant entre elles sur différentes applications, en général afin d'améliorer les performances de classification ou de regroupement de données d'une carte auto-organisatrice classique. Ces travaux se retrouvent sous le terme de SOM hiérarchiques, SOM multi-couches, ou \emph{Deep SOM}.
% Cependant, peu de travaux ont exploré l'aspect topologique et la simplicité des règles de calcul d'une carte pour les assembler en architectures modulaires comportant des rétroactions.
% Nous cherchons en plus à associer leur activité en un système dynamique, conférant à une architecture de cartes un comportement de prise de décision.

% L'étude d'une architecture non hiérarchique de cartes est d'une part motivée par leur inspiration biologique. Leur organisation rappelle en effet celle qu'on peut observer dans des aires cérébrales. Le cortex faisant apparaître des aires interagissant entre elles avec des boucles de rétroaction, la création d'une architecture non hiérarchique de cartes s'inscrit dans la continuité de cette inspiration biologique.
% Ensuite, l'étude des systèmes biologiques et la robotique sont liées~: la biologie sert d'inspiration à la robotique, que ce soit pour le mouvement d'un bras ou la prise de décision, et la robotique permet de tester des théories cherchant à modéliser des comportements biologiques \cite{Oudeyer2010OnTI}.
% Aussi les architectures de cartes bio-inspirées que nous avons relevées dans la littérature se placent aussi dans les domaines des neurosciences computationnelles ou de l'apprentissage incarné (\emph{Embodied intelligence}) en robotique \cite{Smith2005TheDO,cangelosi_embodied_2015}, à la frontière entre étude de la biologie et apprentissage automatique.

% Cette notion d'architecture modulaire de cartes est bien résumée par Kohonen dès 1995~:
% \begin{quote}
% Un objectif à long terme de l'auto-organisation est de créer des systèmes autonomes dont les éléments se contrôlent mutuellement et apprennent les uns des autres. De tels éléments de contrôle peuvent être implémentés par des SOMs spécifiques~; le problème principal est alors l'interface, en particulier la mise à l'échelle automatique des signaux d'interconnexion entre les modules et la collecte de signaux pertinents comme interface entre les modules. Nous laisserons cette idée aux recherches futures.
% \cite{Kohonen1995SelfOrganizingM}
% \end{quote}
% Ces éléments de contrôle sont les modules d'une architecture. 

% L'objectif de nos travaux est ainsi de proposer un modèle de carte qui puisse être utilisée en tant que module, de définir l'interface entre les modules afin de créer une architecture et de comprendre les comportements de calcul qui émergent de l'association des modules.


% Le chapitre 1 présente une zoologie des architectures de cartes de Kohonen existant dans la littérature.



Les travaux présentés dans cette thèse ont fait l'objet de deux présentations en conférence~:

% A placer : 

% Apprentissage supervisé / non supervisé définition.
% Mémoire associative/traitement de séquences 


% \begin{itemize}
%     \item Apprentissage non supervisé, quantification vectorielle : définition + apprentissage développemental ? Pq c'est cool de continuer a etudier les SOM ?
%     \item Modularité et modularité dans les programmes informatiques,définition
%     \item Systèmes dynamiques complexes~: Biologique first puis exemple automates cellulaires qui sont une machine de turing, réseaux de Hopfields, machine de bolztmann: comporements de calcul comme émergeance
%     \item Systèmes d'apprentissage modulaire : des systèmes complexe.
% \end{itemize}

% \cite{Oudeyer2010OnTI} : biologie liée dans les deux sens à l'aspect computationnel.


% But : montrer comment un mécanisme de recherche de consensus entre cartes de Kohonen permet de construire des architectures apprenant des relations multimodales.

% \section{Comment construire un système modulaire d'apprentissage non supervisé}

% La modularité présente des aspects avantageux voire optimaux. 
% Une structure de réseau en petit monde est  utilisée dans des systèmes d'information, comme des bases de données, comme une structure optimisant la vitesse des échanges d'information dans le système. Or, ces structures sont également observées dans de nombreux domaines expérimentaux : biologie (exemple ?) et dans des systèmes sociaux tels que les arbres de connaissances entre individus. 

% D'un point de vue informatique, la modularité possède également de nombreuses définitions.
% On peut définir la modularité comme la décomposition d'un système en sous-systèmes plus petits et fonctionnant indépendamment. Ces systèmes interagissent via une interface bien définie.
% Plusieurs façons de construire ces sous-systèmes.
% Modules ayant des fonctions et structures différentes prédéfinie, exemple en programmation.
% Modules de même structure se différenciant au cours d'une interaction avec l'environnement (apprentissage). Dans ce cas de figure ces composants sont conçus pour être interchangeables et réutilisables.
% La modularité permet alors à un système d'être robuste à un dysfonctionnement d'un composant, sa mise à l'échelle et une faculté d'adaptation.

% Un système modulaire n'est pas forcément un système complexe.
% Cependant, lorsque l'interaction des modules est non linéaire, le système mène à l'émergence de nouveaux comportements.

% \section*{Modularité et Conception d'architectures d'apprentissage}

% Architecture d'apprentissage non supervisé = apprentissage de représentation. Passe par l'encodage de features de l'espace d'entrées sur des systèmes de calcul. 

% Les systèmes d'apprentissage existants combinent la notion de modularité et d'émergence pour former des comportements d'apprentissage plus complexes.
% Les réseaux de deep learning, se sont éloignés du modèle biologique du neurone pour ajouter des règles de calculs plus informatiques comme la backpropagation. Cependant l'approche modulaire est resté une constante dans le développement des réseaux utilisés actuellements comme le modèle teacher student et les réseaux de neurones adversarials qui combinent des réseaux performant chacun une sous-tâche par rapport à l'autre.

% La conception de système modulaire d'apprentissage peut passer par deux approches. D'un côté, une approche “finale” dans laquelle la finalité, l'application du système est connue. La conception du système passe alors par la décomposition de l'objectif en sous-systèmes et sous-tâches de manière à définir des modules particuliers.
% L'approche inverse serait l'approche constructive, dans laquelle nous disposons de modules permettant des comportements simples de calculs et les associons entre eux pour former un système modulaire. Il est difficile de connaître à l'avance le comportement final de ce type de système et son étude passe donc par la simulation.
% Cette approche, si elle n'est pas la plus directe en termes de résultats applicatifs, a l'avantage d'ouvrir la porte à des comportements d'apprentissages qui peuvent être inattendus. 

% Exemples d'archi modulaires d'apprentissage ?

% \subsubsection*{Mémoire associative et modularité}


% \subsection*{ Les cartes auto-organisatrices comme choix de modules : problématique de la thèse }