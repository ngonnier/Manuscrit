\chapter*{Introduction}

Les systèmes biologiques qui nous entourent présentent une incroyable diversité de structures leur permettant d'évoluer et de s'adapter à leur environnement.
Des systèmes en apparence simples présentent ainsi des capacités d'optimisation de tâches. Le blob est par exemple un être unicellulaire capable de s'étendre sur plusieurs mètres. Uniquement grâce aux communications chimiques opérant au sein de son plasmodium, il est capable de trouver le chemin le plus court dans un labyrinthe entre deux points sur lesquels sont placés de la nourriture \cite{Nakagaki2000IntelligenceMB} ou de retrouver une configuration optimale d'un réseau de transport. Il sont même capables d'apprentissage~: deux entités qui fusionnent se transmettent des connaissances de leur environnement, montrant qu'elles ont chacune emmagasiné, à un point, de l'information sur ce dernier.
Les colonies de fourmis, quant à elles constituées de milliers, voire de millions d'individus, sont capables de s'auto-organiser pour effectuer des tâches complexes de coopération pour construire leur nid, se défendre face aux prédateurs et trouver leur nourriture via une communication par leurs phéromones, son et toucher.

Autrement dit, ces systèmes biologiques présentent des capacités de calcul remarquables. 
Toutes ces stratégies mises en place par des systèmes biologiques ont inspiré de nombreux algorithmes d'optimisation imitant par exemple les colonies de fourmis, les essaims d'abeilles, les groupes de chats, les bancs de poissons ou encore les baleines, tous ces groupes d'animaux présentant des méthodes de communication décentralisées efficaces pour accomplir une tâche donnée~\cite{Darwish2018BioinspiredCA}.

Un parangon de système biologique de calcul est sans conteste le cerveau humain ou animal, qui est capable d'exécuter des tâches de calcul et d'apprentissage incroyablement sophistiquées via une multitude de signaux électriques et chimiques circulant entre les neurones et au sein des vaisseaux sanguins.
L'inspiration biologique occupe ainsi une place de premier rang dans les débuts de la recherche en intelligence artificielle.
Face à des modèles d'intelligence artificiels plus symboliques, les modèles à base d'apprentissage ont été initialement développés en s'appuyant sur le modèle biologique du neurone, afin de chercher à imiter les capacités d'évolution et d'adaptation à l'origine de la notion d'apprentissage dans les réseaux de neurones biologiques.
Le perceptron, modèle à l'origine des réseaux de Deep Learning les plus performants à l'heure actuelle, s'inspirait par exemple du modèle de neurone biologique~\cite{McCulloch1990ALC}.
La diversité des applications actuelles de l'apprentissage automatique et le développement de nombreux modèles très performants ont amené la recherche en apprentissage automatique à se concentrer principalement sur une approche algorithmique et computationnelle de ces réseaux, en particulier pour les architectures de Deep Learning. 
Cet éloignement de la biologie permet de s'extraire des contraintes physiques liées au neurone pour envisager des règles de calcul adaptées à une tâche spécifique.
Néanmoins, la biologie, par sa diversité de comportements encore incompris, reste une source d'inspiration abondante pour apporter des paradigmes alternatifs ou complémentaires aux modèles d'apprentissage existants dans l'état de l'art. De plus, le comportement du cerveau est loin d'être entièrement compris et modélisé, ce qui signifie que les possibilités d'inspiration biologique sont constamment en train d'évoluer.
Les réseaux de neurones impulsionnels (\emph{Spiking Neural Networks}) sont un exemple de modèle d'apprentissage illustrant une complémentarité récente entre l'approche bio-inspirée et l'approche computationnelle.
Ces réseaux de neurones ont été développés dès les années 1990~\cite{Maass1996NetworksOS} et s'appuient directement sur le modèle biologique du neurone.
Ils apparaissent dans de nombreux travaux récents comme une méthode montante dans le domaine de l'apprentissage automatique pour la conception de modèles d'apprentissage moins énergivores et distribués, grâce à la conception d'architectures matérielles neuromorphiques telles que LOIHI \footnote{\url{https://www.intel.com/content/www/us/en/research/neuromorphic-computing.html}}. De nombreux travaux cherchent ainsi à adapter des réseaux de neurones classiques de l'état de l'art dans une version impulsionnelle, faisant ainsi passer les SNN de la biologie au calcul~\cite{Schuman2022OpportunitiesFN}.
De tels modèles apportent de nouveaux paradigmes de calcul pouvant se combiner avec des approches plus appliquées.
Nous pensons ainsi qu'il est pertinent de continuer à explorer des modèles d'apprentissage automatique inspirés de la biologie.

La thèse que nous présentons présente un modèle d'apprentissage élaboré à la fois dans une démarche d'inspiration biologique et utilisant des techniques plutôt calculatoires.
Concernant l'inspiration biologique, de nombreuses modélisations générales du cortex cérébral, telle que~\cite{binzegger05, Meunier2009HierarchicalMI,sporns_structure_2013,betzel_multi-scale_2017} proposent que le cortex est une architecture composée de modules auto-organisés.
Ces modules communiquent autour des informations sensorielles collectées par l'organisme. Cette communication est réalisée de façon interne, liant des informations sensorielles et abstraites provenant de différentes parties du cortex et à différentes échelles spatiotemporelles. Enfin, bien qu'une hiérarchie de traitement de l'information apparaisse entre ces modules, certains traitant des entrées sensorielles et d'autres des entrées plus abstraites, de nombreux circuits de rétroactions entre les modules semblent présents à différents niveaux de l'architecture.
La recherche d'architectures cognitives s'inspirant de l'architecture du cortex cérébral est un enjeu de longue date dans la recherche en apprentissage automatique. Il s'agit de développer des réseaux de neurones autonomes, capables de mémoire et de prise de décision de façon non supervisée, qui s'inspirent des architectures modulaires présentes dans le cerveau humain et qui cherchent à imiter certains comportements~\cite{Kotseruba201840YO}.
Le développement de la robotique appelle également à envisager des modèles d'apprentissage directement liés à la perception sensorielle, en s'inspirant des architectures modulaires existant dans le cortex et qui permettent le traitement de cette perception.

Dans cette démarche de création d'architecture inspirée de l'architecture corticale, nous proposons dans cette thèse de construire un modèle d'architecture d'apprentissage modulaire non-hiérarchique.
On entend ici par système modulaire un système composé d'une multiplicité de sous-structures interchangeables qui interagissent entre elles par le biais d'interfaces définies. 
Dans une architecture modulaire, ces sous-structures communiquent de façon locale, sans être supervisées par un processus externe. Cette propriété de modularité est partagée par de nombreux systèmes biologiques et artificiels et présente des avantages en termes de réutilisation, de robustesse aux fautes, de redondance et de traitement local de l'information \cite{clune_evolutionary_2013}.
Pour construire cette architecture modulaire, nous nous appuierons sur un modèle d'apprentissage existant inspiré de la biologie~: les cartes auto-organisatrices (SOM) \cite{Kohonen1982}.
Une carte auto-organisatrice est un ensemble de poids, appartenant à un espace d'entrée, disposés sur une grille. Grâce aux règles d'apprentissage, les poids se disposent de façon à représenter l'espace des entrées qui ont été présentées.
Les cartes de Kohonen sont initialement inspirées des cartes topographiques présentes dans les cortex sensoriels ou moteurs. Les mécanismes de choix du BMU et de mise à jour, d'abord calculés grâce à des champs neuronaux, ont ensuite évolué vers des règles plus computationnelles de calcul de distance entre les entrées et les poids des cartes.
De nombreux travaux ont cherché à associer les SOMs en architecture, que nous passerons en revue dans l'état de l'art de cette thèse, mais peu ont exploré l'aspect topologique et la simplicité des règles de calcul d'une carte pour les assembler en architectures modulaires comportant des rétroactions. C'est pourquoi nous nous intéressons à cet aspect dans cette thèse.

Outre l'inspiration biologique que nous avons présentée, la construction de systèmes d'apprentissage modulaires a également une motivation computationnelle.
Elle découle d'une part de l'observation que la combinaison de principes et d'algorithmes simples en architecture modulaire peut mener au développement de mécanismes de calcul complexes au sein du système. De ce fait cette approche est pertinente pour la conception de systèmes innovants, robustes, etc \cite{chen_modularity_2016}.
Les structures modulaires ont notamment la particularité d'engendrer des comportements émergents dans leur dynamique, c'est-à-dire que le comportement global du système résulte bien de l'interaction entre les modules et non seulement de la somme des comportements des modules pris individuellement~: il s'agit de systèmes complexes.
Dans le cas de systèmes modulaires, le comportement général qui émerge est une organisation des modules de l'architecture~: on parle d'auto-organisation du système. Il peut s'agir d'une synchronisation, d'une collaboration ...
Cette auto-organisation est effectuée de façon décentralisée, via des règles et interactions locales au sein de l'architecture modulaire~: elle n'est pas supervisée par un processus extérieur à l'architecture.
D'autre part, les systèmes modulaires présentent des propriétés de robustesse
\cite{Tang2021ARO}
Cette démarche d'utilisation de systèmes modulaires décentralisés a été mise en \oe{}uvre dans les algorithmes d'intelligence d'essaim \cite{Bonabeau1999SwarmI}. Ces algorithmes nous motivent également à considérer des architectures modulaires pour l'apprentissage.
Dans cette thèse, nous nous plaçons ainsi dans une démarche de construction d'un système d'apprentissage~: l'assemblage de modules ayant des règles simples d'évolution peut mener à des comportements collectifs plus complexes, dans notre cas en vue d'un apprentissage. Nous savons que chaque module, une carte de Kohonen, est capable d'apprendre une représentation de son espace d'entrée. Nous nous attendons à diversifier les comportements d'apprentissage.
\cite{chen_modularity_2016}

Finalement, la conception d'une architecture modulaire d'apprentissage s'inspire d'une part des propriétés de modularité et d'émergence observées dans différents systèmes naturels en particulier dans le cerveau, et artificiels comme l'intelligence d'essaim ou les réseaux de neurones profonds.
La propriété de modularité permettrait à une architecture de réaliser des calculs complexes en exploitant la collaboration et l'interaction des modules de façon décentralisée.
Dans cette démarche, les SOMs sont de bonnes candidates pour être associés en architectures modulaires. Elles sont en effet inspirées de l'organisation des aires corticales, aires qui sont associées dans le cerveau en architecture modulaires. L'introduction de rétroactions dans ces architectures permettra de plus d'explorer de nouveaux comportements computationnels.

\section*{Problématique de la thèse}


- Construction d'une architecture modulaire de cartes auto-organisatrices. 
Cette thèse propose de construire une architecture modulaire de cartes auto-organisatrices. Cette démarche s'appuie sur différents modèles d'architecture développés au sein de l'équipe de recherche.
Cette approche est constructive~: nous définissons un modèle en s'appuyant sur l'état de l'art et les travaux précédents, puis nous cherchons à évaluer son comportement, afin de guider les améliorations ou applications qui en découlent.
La notion d'architecture cognitive présentée plus haut cherche à explorer de grands principes liés à la cognition tels que l'apprentissage autonome, sans supervision, le traitement de données temporelles, l'apprentissage sur le long terme sans oubli catastrophique des données précédentes et la fusion de données multimodales, s'inspirant du traitement multisensoriel du cerveau humain.
Nous voulons ainsi évaluer les mécanismes d'apprentissage de l'architecture que nous proposons sur ces principes de calculs. 
Cette problématique est très vaste.
Nous avons choisi dans cette thèse de nous concentrer sur la tâche particulière d'apprentissage associatif de données multimodales. 
Il s'agit d'apprendre les relations existant entre les entrées, en plaçant cet apprentissage de relations à un niveau interne à l'architecture.
Cette mémoire associative apparaît comme un type d'apprentissage non supervisé. Le principe est d'apprendre une représentation des entrées, mais également de leurs relations.


Ainsi cette thèse présente des travaux visant à identifier les mécanismes d'apprentissage associatifs à l'\oe{}uvre dans des architectures de quelques cartes.
Nous cherchons à répondre aux questions suivantes :
\begin{itemize}
    \item Comment construire une architecture de cartes auto-organisatrices ?
    \item Comment évaluer l'encodage des entrées et de leurs relations au sein d'une architecture de cartes auto-organisatrices ?
    \item Comment l'architecture CxSOM permet un apprentissage associatif de données ?
\end{itemize}


\section*{Contributions et plan}

Le manuscrit est organisé de la façon suivante.
Le chapitre~\ref{chap:archis} présente un état de l'art des architectures de cartes auto-organisatrices existantes dans la littérature, afin de définir ce qu'on entend par architecture modulaire décentralisée et positionner notre modèle dans l'ensemble des modèles existants. Nous nous intéresserons en particulier aux interfaces choisies entre les modules.

Nous détaillerons ensuite le modèle d'architecture décentralisée de cartes auto-organisatrices que nous proposons dans cette thèse, CxSOM (\emph{Consensus-driven Multi-SOM}) au chapitre~\ref{chap:modele}.
Ce modèle d'architecture permet d'associer des cartes en architecture non-hiérarchiques. Dans ce modèle, les activités des différentes cartes sont interdépendantes et l'apprentissage s'appuie sur une recherche de consensus entre les cartes pour la recherche d'un BMU.
Si le modèle a pour but à long terme de concevoir une architecture comportant de nombreux modules ainsi que des connexions temporelles, nous avons concentré cette thèse sur l'analyse des comportements d'architecture de deux et trois cartes.
Le but de cette thèse est de proposer une méthodologie d'analyse de ce modèle et d'en tirer des comportements élémentaires qui serviront à poser les jalons de la construction d'une architecture plus complexes.
Les expériences présentées dans la suite du manuscrit étudient le comportement de ces architectures sous différents angles d'approche.
Le chapitre~\ref{chap:relaxation} présente une analyse plus approfondie du mécanisme de relaxation entre les BMUs, constituant l'interface entre les cartes. 
Ce chapitre cherche à répondre à la pertinence de ce mécanisme en tant qu'algorithme de choix de BMU pour l'apprentissage.

Nous introduisons au chapitre \ref{chap:repr} une méthode expérimentale et des représentations exprimant comment une architecture de cartes encode les entrées et leurs relations.
Nous présentons ensuite au chapitre \ref{chap:analyse} les comportements élémentaires d'apprentissage associatif observés sur des architectures de deux et trois cartes en une dimension, qui sont plus facile à visualiser. 
Nous présenterons en particulier un comportement de prédiction d'entrée rendu possible par le modèle.
Nous proposons au chapitre~\ref{chap:indicateur} des indicateurs numériques originaux d'évaluation de l'apprentissage associatif par l'architecture de cartes, dans le but d'étendre l'analyse du modèle à des architectures difficilement représentables.
Le chapitre \ref{chap:analyse2D} étend enfin les observations aux architectures de cartes en deux dimensions, afin de saisir la scalabilité du modèle.




Idées : plutôt se concentrer sur la SOM en elle-même. Champs d'application, performances, remarquer que les SOM ont des performances pour extraire des représentations, inspi biologique
Voila ce qui n'a pas été réalisé, et voila ce que je propose : injecter de l'information de position 
démarche synthétique
caractériser le comportement émergent des cartes.