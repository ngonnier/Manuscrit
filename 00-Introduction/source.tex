\section*{Introduction}
%\subsection*{Bio-inspiration : une thématique actuelle}

Le terme calcul désigne les procédés abstraits utilisés pour le traitement de l'information. 
Un système réalisant de tels procédés est alors désigné comme système de calcul. 
D'un point de vue abstrait, ces systèmes peuvent ainsi être biologiques comme artificiels.
Aussi de nombreux systèmes biologiques excellent à faire du calcul dès lors qu'ils interagissent avec leur environnement et s'adaptent à ce dernier~: les réseaux mycellaires communiquent des informations au sein du et échangent avec leur environnement~; les essaims d'abeilles sont capables de cartographier leur environnement grâce aux informations échangées entre individus. Enfin, le cerveau est bien sur un système de calcul~: il transforme une multitude de signaux physiques et chimiques provenant des capteurs du corps humain en activité électrique qui se traduisent en des actions sur l'environnement.

Ainsi, de nombreux systèmes artificiels de calcul s'inspirent de la biologie, et en particulier les systèmes d'apprentissage. 
Les modèles d'apprentissage de Deep Learning les plus performants à l'heure actuelle proviennent du perceptron, modèle inspiré des neurones biologiques.
La diversité des applications actuelles de l'apprentissage automatique et les méthodes de développement de ces algorithmes se sont éloignées d'une recherche de compréhension de la biologie. 
Cependant, la biologie, par sa diversité de comportements encore incompris, reste une inspiration majeure dans le domaine des réseaux de neurones et peut encore apporter des paradigmes différents.
Par exemple, les réseaux de neurones impulsionnels (\emph{Spiking Neural Networks}), directement inspirés du modèle biologique de neurones, sont vus comme des "neurones de troisième génération" et sont une méthode montante pour la conception de modèles d'apprentissage moins énergivore.
Enfin, la question des calculs à l'\oe{}uvre dans les systèmes biologique reste une thématique de recherche actuelle dans de nombreux domaines cherchant à simuler des comportements biologiques.-Inversement, les systèmes  de calcul biologiques, par leur diversité de comportements encore incompris, restent une source d'inspiration majeure pour le développement de nouveaux systèmes de calcul artificiels.
Pour \cite{Oudeyer2010OnTI} par exemple, la biologie reste ainsi liée dans les deux sens aux modèles artificiels de calcul.

% Exemples de systèmes bio-inspirés : 
% Colonies de fourmis
% Essaim d'abeille, particle swarm
% Algorithmes génétiques
% Bat algorithmes.
% Spiking neural networks directement inspirés des réseaux de neurones biologiques et sont une thématique de recherche actuelle, par l'avantage énergetique qu'apporte la décentralisation du calcul à l'échelle du neurone.

\subsection*{Modularité des systèmes biologiques}

La modularité des systèmes est une caractéristique récurrente, voire de fond dans les systèmes de calcul biologiques.
La modularité d'un système se définit comme sa composition en sous-systèmes autonomes effectuant des tâches différentes et collaborant entre eux. 
Cette modularité présente des avantages de réutilisation, de robustesse à des fautes, de traitement local de l'information. 

\cite{clune_evolutionary_2013}


La notion de modularité est très générale. Nous pouvons en extraire plusieurs aspects. Des systèmes modulaires peuvent être composés de composants de structure différente interagissant entre eux. Un ordinateur est par exemple composé d'une multitude de modules spécifiques, conçus pour des tâches séparées.
Inversement, un système modulaire peut également être composé de modules de même structure interagissant entre eux par des interfaces. Par exemple, les colonies de fourmi sont composées d'individus de même structure.
Les modules ont une fonction spécifique, qui peut être fournie à priori (fourmis guerrières/ouvrières) ou apprises par le module au cours de son interaction avec l'environnement.
Ainsi, même si le cerveau possède une structure a priori, le rôle de certaines aires cérébrales peut être redéfini au cours du temps~: l'aire visuelle d'un individu aveugle est réorganisée pour effectuer le traitement des autres sens de l'individu. La modularité apporte alors une flexibilité au système.

\subsubsection*{Systèmes dynamiques complexes, emergeance et auto-organisation}

Dans ce type de systèmes modulaires, le comportement dynamique global du système est souvent plus complexe que le comportement de chaque individu. Lorsque c'est le cas, on parle de système complexe et le comportement global un comportement \emph{émergeant} du système modulaire. Cette notion d'émergence, dont nous avons donné des exemples biologiques, a inspiré de nombreux modèles de calcul en intelligence artificielle et robotique.
La fascination pour les comportements émergeant des systèmes complexes vient du fait que ce comportement global n'est analytiquement pas prévisible à partir des données des comportements élémentaires.
Un exemple est l'étude des automates cellulaires dans le jeu de la vie. 
Les règles élémentaires simples 
Par exemple, l'intelligence d'essaim (swarm intelligence) est une application directe de ce concept à des sytèmes multi-agents qui interagissent localement avec leur environnement. 

Les neurones du cerveau en sont un autre exemple. Chaque neurone est construit sur le même modèle et les règles d'évolution de chaque neurone sont similaires. Dans son ensemble, le cerveau présente des comportements de calculs qui lui sont propres.
Dans ce même cerveau, à  une plus grande échelle, on peut séparer des zones fonctionnelles distinctes dans le cerveau. Chaque zone est de construction similaire, mais effectue une fonction différente.
Le cerveau n'est pas le seul système biologique présentant ce type de fonctionnemment. Ainsi, les systèmes métaboliques ou d'expression de gènes sont des exemples de systèmes aggrégeant des éléments.
Les colonies de fourmi sont constituées de milliers d'individus effectuants des actions à leur échelle et communiquant localement. Le comportement de la colonie est un système de calcul, capable de résoudre des problème d'optimisation de chemin vers une source de nourriture.

Les réseaux biologiques hautement modulaires présentant cette capacité de traitement de l'information présentent également des propriétés d'auto-organisation; \cite{Siebert2020RoleOM} suggère même que la modularité est un élément clé pour la présence de motifs auto-organisés. 

\subsubsection{Construire un système modulaire d'apprentissage non supervisé}

La modularité présente des aspects avantageux pour le calcul, voire optimaux. En témoignent les architectures de modèles artificiels, a priori pas du tout inspirés de la biologie qui sont comparables à des modèles biologiques existants. 
Une structure de réseau en petit monde est  utilisée dans des systèmes d'information, comme des bases de données, comme une structure optimisant la vitesse des échanges d'information dans le système. Or, ces structures sont également observées dans de nombreux domaines expérimentaux : biologie (exemple ?) et dans des systèmes sociaux tels que les arbres de connaissances entre individus. 

D'un point de vue informatique,la modularité possède également de nombreuses définitions.
On peut définir la modularité comme la décomposition d'un système en sous-systèmes plus petits et fonctionnant indépendamment. Ces systèmes interagissent via une interface bien définie.
Plusieurs façons de construire ces sous-systèmes.
Modules ayant des fonctions et structures différentes prédéfinie, exemple en programmation.
Modules de même structure se différenciant au cours d'une interaction avec l'environnement (apprentissage). Dans ce cas de figure ces composants sont conçus pour être interchangeables et réutilisables.
La modularité permet alors à un système d'être robuste à un dysfonctionnement d'un composant, sa mise à l'échelle et une faculté d'adaptation.

Un système modulaire n'est pas forcément un système complexe.
Cependant, lorsque l'interaction des modules est non linéaire, le système mène à l'émergence de nouveaux comportements.

\subsection*{Modularité et Conception d'architectures d'apprentissage}

Architecture d'apprentissage non supervisé = apprentissage de représentation. Passe par l'encodage de features de l'espace d'entrées sur des systèmes de calcul. 

Les systèmes d'apprentissage existants combinent la notion de modularité et d'émergence pour former des comportements d'apprentissage plus complexes.
Les réseaux de deep learning, se sont éloignés du modèle biologique du neurone pour ajouter des règles de calculs plus informatiques comme la backpropagation. Cependant l'approche modulaire est resté une constante dans le développement des réseaux utilisés actuellements comme le modèle teacher student et les réseaux de neurones adversarials qui combinent des réseaux performant chacun une sous-tâche par rapport à l'autre.

La conception de système modulaire d'apprentissage peut passer par deux approches. D'un côté, une approche “finale” dans laquelle la finalité, l'application du système est connue. La conception du système passe alors par la décomposition de l'objectif en sous-systèmes et sous-tâches de manière à définir des modules particuliers.
L'approche inverse serait l'approche constructive, dans laquelle nous disposons de modules permettant des comportements simples de calculs et les associons entre eux pour former un système modulaire. Il est difficile de connaître à l'avance le comportement final de ce type de système et son étude passe donc par la simulation.
Cette approche, si elle n'est pas la plus directe en termes de résultats applicatifs, a l'avantage d'ouvrir la porte à des comportements d'apprentissages qui peuvent être inattendus. 

Exemples d'archi modulaires d'apprentissage ? ART, Reservoir

\subsubsection*{Mémoire associative et modularité}


\subsection*{ Les cartes auto-organisatrices comme choix de modules : problématique de la thèse }

Nous avons choisi dans cette thèse de s'intéresser à cette deuxième approche~: développer un système modulaire apprenant en associant des réseaux existants connus. Nous nous intéressons spécialement aux cartes auto-organisatrices.

Si nous revenons à un aspect biologique, les cartes auto-organisatrices sont, par leur comportement un modèle simplifié des aires cérébrales. Les travaux conduits dans notre équipe ces dernières années se sont attachés à construire des architectures modulaires complètement cellulaires. Nous cherchons dans cette thèse à s'inspirer de ces travaux mais en les passant à une échelle moins cellulaire, dans un cadre de simplification du modèle. Cette simplification nous permet une recherche plus facile et moins coûteuse de nouveaux comportements d'apprentissage, tout en restant déclinable si besoin en version cellulaire.
Dans un cadre d'architecture modulaire, 


}




\subsection*{Contributions et plan}

Cette thèse cherche donc, dans un cadre bio-inspiré, à construire une architecture modulaire décentralisée de cartes auto-organisatrices. L'idée de cette approche est de rechercher des nouveaux comportements d'apprentissage émergeant de l'interaction entre les modules d'une grande architecture, à l'inverse des méthodes plus ingénieures consistant à diviser une tâche connue en sous-systèmes.
Nous commencerons par présenter un état de l'art des architectures de cartes auto-organisatrices existantes afin de définir ce qu'on entend par architecture modulaire décentralisée et positionner notre modèle dans l'ensemble des modèles existants.
Nous détaillerons ensuite notre modèle d'architecture décentralisée de cartes auto-organisatrices.
Si le modèle a pour but à long terme de concevoir une architecture comportant de nombreux modules, nous avons concentré cette thèse sur l'analyse des comportements d'architecture de deux et trois cartes.
Le but de cette thèse est alors de proposer une méthodologie d'analyse de ce modèle et d'en tirer des comportements élémentaires.
Nous proposerons une méthode expérimentale et des représentations rapprochant l'architecture de cartes de modèles d'apprentissage communs au chapitre 3.
Les résultats présentés dans les chapitres 4,5,6,7 présentent le comportement du modèle CxSOM sous différents angles.
Nous analyserons plus en détail l'interface entre cartes et une recherche de BMU originale que nous utilisons.
Nous présenterons ensuite les comportements élémentaires observés sur des architectures de deux et trois cartes en une dimension, qui sont plus facile à visualiser. Nous présenterons notamment un comportement de prédiction rendu possible par le modèle.
Nous proposons au chapitre 6 des indicateurs numériques originaux d'évaluation de l'apprentissage associatif par l'architecture de cartes, dans le but d'étendre l'analyse du modèle à des architectures difficilement représentables visuellement.
Le chapitre 7 applique la méthode d'observation à des cartes en deux dimensions afin de saisir la scalabilité du modèle.

Les travaux présentés dans cette thèse ont fait l'objet de deux présentations en conférence~:
\begin{itemize}
    \item Consensus driven ...., ICONIP 2020
    \item Input prediction in SOMs, ISCMI 2021
\end{itemize}

A placer : 

Apprentissage supervisé / non supervisé définition.
Mémoire associative/traitement de séquences 


% \begin{itemize}
%     \item Apprentissage non supervisé, quantification vectorielle : définition + apprentissage développemental ? Pq c'est cool de continuer a etudier les SOM ?
%     \item Modularité et modularité dans les programmes informatiques,définition
%     \item Systèmes dynamiques complexes~: Biologique first puis exemple automates cellulaires qui sont une machine de turing, réseaux de Hopfields, machine de bolztmann: comporements de calcul comme émergeance
%     \item Systèmes d'apprentissage modulaire : des systèmes complexe.
% \end{itemize}

% \cite{Oudeyer2010OnTI} : biologie liée dans les deux sens à l'aspect computationnel.


% But : montrer comment un mécanisme de recherche de consensus entre cartes de Kohonen permet de construire des architectures apprenant des relations multimodales.