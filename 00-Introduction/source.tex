\chapter*{Introduction générale}

%\subsection*{Réseaux de neurones artificiels et inspiration biologique}

Les systèmes biologiques qui nous entourent présentent une incroyable diversité de structures leur permettant d'évoluer et de s'adapter à leur environnement.
Des systèmes en apparence simples présentent ainsi des comportements d'optimisation. Le blob est par exemple un être unicellulaire capable de s'étendre sur plusieurs mètres. Uniquement grâce aux communications chimiques opérant au sein de la membrane, il est capable de trouver le chemin le plus court dans un labyrinthe entre deux points sur lesquels sont placés de la nourriture \cite{Nakagaki2000IntelligenceMB}.
Les colonies de fourmis, quant à elles constituées de milliers, voire de millions d'individus, sont capables de s'auto-organiser pour effectuer des tâches complexes de coopération pour construire leur nid, se défendre face aux prédateurs et trouver leur nourriture via une communication par leurs phéromones, son et toucher \cite{jackson_communication_2006}.
Autrement dit, ces systèmes biologiques présentent des capacités de calcul remarquables, le calcul étant en effet défini comme la propriété de traiter et d'exploiter l'information reçue de son environnement.
Ces stratégies mises en place par des systèmes biologiques ont inspiré de nombreux algorithmes d'optimisation s'inspirant des colonies de fourmis, des essaims d'abeilles, des groupes de chats, des bancs de poissons, des baleines, ou encore des poules.

Un parangon de système biologique de calcul est sans conteste le cerveau humain ou animal, qui est capable d'exécuter des tâches de calcul et d'apprentissage incroyablement sophistiquées via une multitude de signaux électriques et chimiques circulant entre les neurones et au sein des vaisseaux sanguins.
L'inspiration biologique occupe une place de choix dans les débuts de la recherche en intelligence artificielle.
Face à des modèles symboliques, les modèles à base d'apprentissage ont été initialement développés en s'appuyant sur le modèle biologique du neurone, afin de chercher à imiter les capacités d'évolution et d'adaptation à l'origine de la notion d'apprentissage dans les réseaux de neurones biologiques.
Le perceptron, modèle à l'origine des réseaux de Deep Learning les plus performants à l'heure actuelle, s'inspirait par exemple du modèle de neurone biologique~\cite{McCulloch1990ALC}.
La diversité des applications actuelles de l'apprentissage automatique et le développement de nombreux modèles très performants ont amené la recherche en apprentissage automatique à se concentrer principalement sur une approche algorithmique et computationnelle de ces réseaux, en particulier pour les architectures de Deep Learning. 
Cet éloignement de la biologie permet de s'extraire des contraintes physiques liées au neurone pour envisager des règles de calcul adaptées à une tâche spécifique.


Cependant, la biologie, par sa diversité de comportements encore incompris, reste une source d'inspiration abondante pour apporter des paradigmes alternatifs ou complémentaires aux modèles d'apprentissage existants dans l'état de l'art.
Les réseaux de neurones impulsionnels (\emph{Spiking Neural Networks}) sont un exemple de modèle d'apprentissage illustrant une complémentarité récente entre l'approche bio-inspirée et l'approche computationnelle.
Ces réseaux de neurones ont été développés dans les années 1990~\cite{Maass1996NetworksOS} et s'appuient directement sur le modèle biologique de neurone. %DEF
Ils apparaissent pourtant dans de nombreux travaux récents comme une méthode montante dans le domaine de l'apprentissage automatique pour la conception de modèles d'apprentissage moins énergivores et distribués, grâce à des architectures matérielles neuromorphiques telles que LOIHI \footnote{\url{https://www.intel.com/content/www/us/en/research/neuromorphic-computing.html}}. De nombreux travaux cherchent ainsi à adapter des réseaux de neurones classiques de l'état de l'art dans une version impulsionnelle \cite{Schuman2022OpportunitiesFN}.

Se plaçant dans cette démarche d'inspiration biologique, le concept d'architectures cognitives est un enjeu de longue date dans la recherche en apprentissage automatique. Le but est de développer des réseaux de neurones autonomes, capables de mémoire et de prise de décision de façon non supervisée, qui s'inspirent des architectures modulaires présentes dans le cerveau humain \cite{Kotseruba201840YO}.
Le développement de la robotique appelle également à envisager des modèles d'apprentissage directement liés à la perception sensorielle.
Des avancées dans le domaine des architectures inspirées de la cognition peuvent ainsi permettre d'apporter de nouveaux horizons aux modèles d'apprentissage existants.

Enfin, le comportement du cerveau est loin d'être entièrement compris et modélisé, ce qui signifie que les possibilités d'inspiration biologique sont constamment en train d'évoluer.
Cette thèse s'inscrit dans cette démarche d'inspiration biologique, étudiée en complément d'une approche computationnelle, en cherchant à proposer des mécanismes de calculs alternatifs s'inspirant des architectures cérébrales à partir de modèles pré-établis.

Nous avons choisi de nous intéresser en tant que modules à des cartes auto-organisatrices de Kohonen, ou SOM \cite{Kohonen1982}.
Les SOM sont constituées d'une grille de neurone, dont à chaque n\oe{}ud est assigné un poids.
L'apprentissage d'une SOM sur des données consiste à mettre à jour ces poids afin de représenter l'organisation de l'espace d'entrée sur les poids de cette grille 2D.
Pour cela, les règles d'évolution reposent sur un principe de compétition \emph{Winner Take All}, permettant de choisir un poids (BMU) représentant le mieux une entrée. La mise à jour est ensuite effectuée en collaboration entre neurones~: le gagnant est mis à jour ainsi que ses voisins proches. Cette dualité compétition/collaboration entraîne une organisation finale de la carte sur l'espace d'entrée ordonnée~: deux n\oe{}uds proches ont des poids proches, ainsi que conservant d'autres propriétés topologiques.
Les cartes de Kohonen sont initialement inspirées des cartes topographiques présentes dans les cortex sensoriels ou moteurs~; les mécanismes de choix du BMU et de mise à jour, d'abord calculés grâce à des champs neuronaux, ont ensuite évolué vers des règles plus computationnelles de calcul de distance et d'argmin sur la carte.


L'étude d'une architecture non hiérarchique de cartes auto-organisatrices est d'une part motivée par leur inspiration biologique. Leur organisation rappelle en effet celle qu'on peut observer dans des aires cérébrales. Le cortex faisant apparaître des aires interagissant entre elles avec des boucles de rétroaction, la création d'une architecture non hiérarchique de cartes s'inscrit dans la continuité de cette inspiration biologique.
Par ailleurs, de nombreux travaux ont cherché à associer les SOMs en architecture, mais peu ont exploré l'aspect topologique et la simplicité des règles de calcul d'une carte pour les assembler en architectures modulaires comportant des rétroactions. 
Les travaux précédents réalisés dans notre équipe de recherche ont associé des champs neuronaux dynamiques couplés entre eux à des SOMs afin d'ajouter un aspect d'apprentissage à ces réseaux. Les DNFs sont très proches des SOMs par leur notion de voisinage dans le calcul de l'activation et le mécanisme de Winner Take All en résultant, remplacé par un argmax dans une carte de Kohonen. Le couplage rétroactif entre les SOMs dans ces travaux engendre une réponse dynamique des DNFs qui évoluent pour se stabiliser vers une position de consensus pour définir les BMUs des cartes.
La thèse vise donc à utiliser les SOMs en tant que modules d'une architecture, en introduisant un mécanisme de relaxation inspiré des DNFs pour appuyer l'apprentissage sur une notion de consensus dans l'architecture. 
Cette approche permettrait de créer des architectures modulaires avec des propriétés émergentes. 
Ainsi, les SOMs sont de bonnes candidates pour être associés en architectures modulaires, et l'introduction de rétroactions et de connexions temporelles pourrait permettre d'explorer de nouveaux comportements.

-----------------------------------------------------------------------------------------------------

Parmi les mécanismes d'inspiration biologique nous nous intéressons à la modularité et l'auto-organisation.
Nous voyons ici la modularité comme la décomposition d'un système en une multiplicité de sous-structures interagissant entre elles parle biais d'interface définies.
Globalement, cette propriété de modularité est partagée par de nombreux systèmes biologiques et 
présente des avantages de réutilisation, de robustesse aux fautes, de redondance et de traitement local de l'information.
Les structures modulaires présentes en biologie sont fondamentalement non-hiérarchiques~: aucun processus supérieur n'\oe{}uvre à superviser leur comportement. Ainsi, 
Le cortex cérébral peut par exemple être décomposé, fonctionnellement ou anatomiquement, en modules interagissant entre eux.

La motivation de s'intéresser à des systèmes modulaires vient de leur dynamique complexe. De tels systèmes présentent un comportement global qui résulte de l'interaction entre les modules~: il s'agit d'un comportement émergeant. En général, chaque module a des règles d'évolution simples, mais l'assemblage des modules peut mener à une dynamique complexe.
D'un point de vue biologique, l'apprentissage à l'\oe{}uvre dans le cerveau peut ainsi être défini comme un comportement émergent. Chaque neurone a des règles d'évolution simples, mais le comportement global du système résulte de l'interaction entre neurones. 
L'émergence d'un comportement au sein de systèmes modulaires est intimement liée à la notion d'auto-organisation.
???

D'un point de vue computationnel, l'émergence de comportements au sein d'architectures modulaires est une piste intéressante pour développer de nouvelles architectures et a été exploitée pour concevoir des réseaux utilisés aujourd'hui. Un exemple célèbre d'émergence sont les automates cellulaires du jeu de la vie de Conway \cite{Gardener1970MathematicalGT}. Il s'agit de cellules pouvant prendre un état binaire, comportant des règles d'évolution temporelles très simples basées sur les états voisins~; cependant, des structures très complexes et organisées émergent de leur comportement collectif.
Au niveau des systèmes d'apprentissage, cette notion d'émergence apparaît également par exemple dans les réseaux de Deep Learning, que l'on peut voir comme un assemblage de couches permettant chacune un apprentissage.
Sumsumption de Brooks 
architectures de DNF couplés 

Modularité, par son inspiration biologique et ses propriétés en tant que systèmes dynamiques, peut apporter des comportements nouveaux au sein de systèmes d'apprentissage.
Cette thèse se place dans une démarche constructive~: nous chercherons à construire une architecture modulaire en définissant des modules s'interfacants entre eux, et verrons comment leur comportement collectif peur 

La notion d'architecture cognitive présentée plus haut cherche à explorer de grands principes liés à la cognition tels que l'apprentissage autonome, sans supervision, le traitement de données temporelles, l'apprentissage sur le long terme sans oubli catastrophique des données précédentes et la fusion de données multimodales, s'inspirant du traitement multisensoriel du cerveau humain.

L'objectif de nos travaux est ainsi de proposer un modèle de carte qui puisse être utilisée en tant que module, de définir l'interface entre les modules afin de créer une architecture et de comprendre les comportements de calcul qui émergent de l'association des modules. 

\section*{Problématique de la thèse}

Ces méthodes interdisciplinaire, entre inspiration biologique et approche computationnelle, nous ont amenés à proposer un modèle d'architecture non-hiérarchique de cartes de Kohonen, CxSOM. Nous travaux cherchent à répondre au problème suivant~: quels sont les comportements d'apren

\section*{Contributions et plan}

Cette thèse cherche donc, dans un cadre bio-inspiré, à construire une architecture modulaire décentralisée de cartes auto-organisatrices. L'idée de cette approche est de rechercher des nouveaux comportements d'apprentissage émergeant de l'interaction entre les modules d'une grande architecture, à l'inverse des méthodes plus ingénieures consistant à diviser une tâche connue en sous-systèmes.
Le manuscrit est organisé de la façon suivante.
Le chapitre~\ref{chap:archis} présente un état de l'art des architectures de cartes auto-organisatrices existantes afin de définir ce qu'on entend par architecture modulaire décentralisée et positionner notre modèle dans l'ensemble des modèles existants.
Nous détaillerons ensuite le modèle d'architecture décentralisée de cartes auto-organisatrices que nous proposons,CxSOM (\emph{Consensus-driven Multi-SOM}).
Ce modèle permet d'associer des cartes en architecture non-hiérarchiques. Dans ce modèle, les activités des cartes sont interdépendantes et l'apprentissage s'appuie sur une recherche de consensus entre les cartes pour la recherche d'un BMU, par un processus dynamique inspiré de la relaxation entre DNF.
Si le modèle a pour but à long terme de concevoir une architecture comportant de nombreux modules, nous avons concentré cette thèse sur l'analyse des comportements d'architecture de deux et trois cartes.
Le but de cette thèse est de proposer une méthodologie d'analyse de ce modèle et d'en tirer des comportements élémentaires.
Les expériences présentées dans la suite du manuscrit étudient le comportement de ces architectures sous différents points de vue.
Le chapitre~\ref{chap:relaxation} présente une analyse plus approfondie du mécanisme de relaxation permettant la recherche de BMUs entre cartes.
Nous proposerons ensuite une méthode expérimentale et des représentations rapprochant l'architecture de cartes de modèles d'apprentissage communs au chapitre \ref{chap:repr}.
Nous présenterons ensuite les comportements élémentaires observés sur des architectures de deux et trois cartes en une dimension, qui sont plus facile à visualiser. 
Nous présenterons notamment un comportement de prédiction rendu possible par le modèle.
Nous proposons au chapitre~\ref{chap:indicateur} des indicateurs numériques originaux d'évaluation de l'apprentissage associatif par l'architecture de cartes, dans le but d'étendre l'analyse du modèle à des architectures difficilement représentables visuellement.
Le chapitre \ref{chap:analyse2D} utilise la méthodologie pour analyser le comportement d'architectures de cartes en deux dimensions afin de saisir la scalabilité du modèle.












% Le terme calcul désigne les procédés abstraits utilisés pour le traitement de l'information. 
% Un système réalisant de tels procédés est alors désigné comme système de calcul. 
% D'un point de vue abstrait, ces systèmes peuvent ainsi être biologiques comme artificiels.
% Aussi de nombreux systèmes biologiques excellent à faire du calcul dès lors qu'ils interagissent avec leur environnement et s'adaptent à ce dernier~: 
% les réseaux mycellaires communiquent des informations au sein du réseau et échangent avec leur environnement~; les essaims d'abeilles sont capables de cartographier leur environnement grâce aux informations échangées entre individus. 
% Le blob est un être unicellulaire capable de s'étendre, qui est capable de résoudre des problèmes de recherche de plus court chemin, et ce seulement à partir de signaux chimiques circulant au sein de la cellule.
% Enfin, le cerveau est bien entendu un système de calcul~: il transforme une multitude de signaux physiques et chimiques provenant des capteurs du corps humain en activité électrique, dont émerge un apprentissage et des actions sur son environnement.
% Cette capacité d'apprentissage et surtout la recherche de son imitation est à l'origine du développement de l'intelligence artificielle et en particulier de l'apprentissage automatique, désignant les systèmes capable d'apprendre et de généraliser des informations à partir des données qui leur sont présentées.
% Les modèles d'apprentissage de Deep Learning les plus performants à l'heure actuelle s'appuient sur le perceptron, modèle inspiré à l'origine des neurones biologiques.
% La diversité des applications actuelles de l'apprentissage automatique et les méthodes de développement de ces algorithmes ont amené la recherche actuelle à se concentrer principalement sur les problèmes algorithmiques et computationnels que sur le développement de réseaux d'inspiration biologique.
% Cependant, la biologie, par sa diversité de comportements encore incompris, peut rester une inspiration dans le domaine des réseaux de neurones et apporte des paradigmes alternatifs pour la conception de structures d'apprentissage.
% Un exemple récent montrant le lien entre l'approche computationnelle et l'approche bio-inspirée se trouve dans les réseaux de neurones impulsionnels ou\emph{Spiking Neural Networks}. Ces réseaux sont directement inspirés du modèle biologique du neurone, et leur développement remonte à la fin des années 1980~\cite{Maass1996NetworksOS}.
% Ils apparaissent pourtant dans de nombreux travaux récents comme une méthode montante dans le domaine de l'apprentissage automatique pour la conception de modèles d'apprentissage moins énergivores et distribués. De nombreux travaux sur ces modèles de réseaux de neurones cherchent maintenant à adapter des réseaux de neurones classiques dans une version impulsionnelle. Il s'agit ici d'un exemple dans lequel l'inspiration biologique donne des alternatives possibles aux réseaux de neurones actuels et associent de façon complémentaire une approche bio-inspirée une approche plus computationnelle.
% D'un autre côté, les mécanismes à l'\oe{}uvre dans le cerveau restent loin d'être compris et modélisés.
% La question des calculs à l'\oe{}uvre dans les systèmes biologiques reste ainsi une thématique de recherche actuelle. 
% En ce sens, les découvertes de ces domaines sont des sources d'inspiration biologiques pertinentes.
% L'inspiration biologique cherche également à pl
% Nous plaçons cette thèse dans cette dynamique de recherche de réseaux de neurones inspirés de la biologie. Le but de ce domaine est d'étudier des mécanismes de calculs alternatifs qui pourront être intégrés et combinés à des approches plus classiques de conception de systèmes d'apprentissage, ou s'appliquer sur des applications robotiques.
% Une propriété globale des systèmes biologique tient dans la modularité et la non-hiérarchie des systèmes.
% On entend, par modularité d'un système sa composition en sous-systèmes autonomes effectuant des tâches différentes et collaborant entre eux. 
% Les systèmes modulaires présentent des avantages de réutilisation, de robustesse aux fautes, de redondance et de traitement local de l'information. Cette construction se retrouve dans les systèmes biologiques, \cite{clune_evolutionary_2013} proposant que ces avantages apportés par la modularité ont favorisé cette propriété lors de l'évolution.
% Cette de modularité est très générale. Nous pouvons en définir plusieurs aspects. 
% Des systèmes modulaires peuvent être composés de composants de structure différente interagissant entre eux. Un ordinateur est par exemple composé d'une multitude de modules spécifiques, conçus pour des tâches séparées et de structures complètement différentes.
% Un système modulaire peut également être composé de modules de même structure interagissant entre eux par des interfaces. 
% Par exemple, les colonies de fourmis sont composées d'individus de même structure.
% Les modules ont une fonction spécifique, qui peut être fournie a priori (fourmis guerrières/ouvrières) ou apprises par le module au cours de son interaction avec l'environnement.
% Si on s'intéresse au cerveau, ce dernier possède une structure a priori, mais le rôle de certaines aires cérébrales peut être redéfini et réappris au cours du temps~: l'aire visuelle d'un individu aveugle est réorganisée pour effectuer le traitement des autres sens de l'individu. 
% La modularité apporte alors une flexibilité au système.

% \subsection*{L'auto-organisation, une propriété bio-inspirée}

% Dans certains systèmes modulaires, le comportement dynamique global du système est souvent plus complexe que le comportement de chaque individu. 
% Lorsque c'est le cas, on parle de système complexe et le comportement global un comportement \emph{émergent} du système modulaire. Cette notion d'émergence, dont nous avons donné des exemples biologiques, a inspiré de nombreux modèles de calcul en intelligence artificielle et robotique.
% La fascination pour les comportements émergeant des systèmes complexes vient du fait que ce comportement global n'est analytiquement pas prévisible à partir des données des comportements élémentaires.
% Un exemple computationnel est l'étude des automates cellulaires dans le jeu de la vie. 
% Malgré des règles élémentaires simples, les automates présentes des capacités de mémoire et de transmission de donnée.
% L'intelligence d'essaim (swarm intelligence) est également une application directe du concept d'émergence à des sytèmes multi-agents qui interagissent localement avec leur environnement. 

% Ici encore, la cognition apparaît comme un comportement émergent de l'activité des neurones. Chaque neurone est construit sur le même modèle et les règles d'évolution de chaque neurone sont similaires. 
% Dans son ensemble, le cerveau présente des comportements de calculs qui lui sont propres.
% Dans ce même cerveau, à  une plus grande échelle, on peut séparer des zones fonctionnelles distinctes dans le cerveau. Chaque zone est de construction similaire, mais effectue une fonction différente.
% Le cerveau n'est pas le seul système biologique présentant ce type de fonctionnemment. Ainsi, les systèmes métaboliques ou d'expression de gènes sont des exemples de systèmes aggrégeant des éléments.
% Les colonies de fourmi sont constituées de milliers d'individus effectuants des actions à leur échelle et communiquant localement. Le comportement de la colonie est un système de calcul, capable de résoudre des problème d'optimisation de chemin vers une source de nourriture.

% Les réseaux biologiques hautement modulaires présentant cette capacité de traitement de l'information présentent également des propriétés d'auto-organisation; \cite{Siebert2020RoleOM} suggère même que la modularité est un élément clé pour la présence de motifs auto-organisés. 

% L'apprentissage est un exemple de comportement émergent d'un système. Perceptron multicouches comme comportement émergent de plusieurs couches.
% D'un point de vue modulaire, un exemple de système  = architectures de champs neuronaux dynamiques.
% Apparition de comportements complexes \cite{Sandamirskaya2014DynamicNF}


% \subsection*{Les cartes auto-organisatrices, un module bio-inspiré}

% Nous avons choisi dans cette thèse de s'intéresser à cette deuxième approche~: développer un système modulaire apprenant en associant des réseaux existants connus. 
% Nous nous intéressons spécialement aux cartes auto-organisatrices.

% Si nous revenons à un aspect biologique, les cartes auto-organisatrices sont, par leur comportement un modèle simplifié des aires cérébrales. Les travaux conduits dans notre équipe ces dernières années se sont attachés à construire des architectures modulaires complètement cellulaires. Nous cherchons dans cette thèse à s'inspirer de ces travaux mais en les passant à une échelle moins cellulaire, dans un cadre de simplification du modèle. Cette simplification nous permet une recherche plus facile et moins coûteuse de nouveaux comportements d'apprentissage, tout en restant déclinable si besoin en version cellulaire.
% Dans un cadre d'architecture modulaire,

% Les cartes auto-organisatrices et notamment le modèle de Kohonen sont largement utilisées en tant qu'algorithme d'apprentissage non supervisé appliqué à des tâches de réduction de dimension, de visualisation de données ou de classification.
% De nombreux travaux étudient l'utilisation de plusieurs cartes collaborant entre elles sur différentes applications, en général afin d'améliorer les performances de classification ou de regroupement de données d'une carte auto-organisatrice classique. Ces travaux se retrouvent sous le terme de SOM hiérarchiques, SOM multi-couches, ou \emph{Deep SOM}.
% Cependant, peu de travaux ont exploré l'aspect topologique et la simplicité des règles de calcul d'une carte pour les assembler en architectures modulaires comportant des rétroactions.
% Nous cherchons en plus à associer leur activité en un système dynamique, conférant à une architecture de cartes un comportement de prise de décision.

% L'étude d'une architecture non hiérarchique de cartes est d'une part motivée par leur inspiration biologique. Leur organisation rappelle en effet celle qu'on peut observer dans des aires cérébrales. Le cortex faisant apparaître des aires interagissant entre elles avec des boucles de rétroaction, la création d'une architecture non hiérarchique de cartes s'inscrit dans la continuité de cette inspiration biologique.
% Ensuite, l'étude des systèmes biologiques et la robotique sont liées~: la biologie sert d'inspiration à la robotique, que ce soit pour le mouvement d'un bras ou la prise de décision, et la robotique permet de tester des théories cherchant à modéliser des comportements biologiques \cite{Oudeyer2010OnTI}.
% Aussi les architectures de cartes bio-inspirées que nous avons relevées dans la littérature se placent aussi dans les domaines des neurosciences computationnelles ou de l'apprentissage incarné (\emph{Embodied intelligence}) en robotique \cite{Smith2005TheDO,cangelosi_embodied_2015}, à la frontière entre étude de la biologie et apprentissage automatique.

% Cette notion d'architecture modulaire de cartes est bien résumée par Kohonen dès 1995~:
% \begin{quote}
% Un objectif à long terme de l'auto-organisation est de créer des systèmes autonomes dont les éléments se contrôlent mutuellement et apprennent les uns des autres. De tels éléments de contrôle peuvent être implémentés par des SOMs spécifiques~; le problème principal est alors l'interface, en particulier la mise à l'échelle automatique des signaux d'interconnexion entre les modules et la collecte de signaux pertinents comme interface entre les modules. Nous laisserons cette idée aux recherches futures.
% \cite{Kohonen1995SelfOrganizingM}
% \end{quote}
% Ces éléments de contrôle sont les modules d'une architecture. 

% L'objectif de nos travaux est ainsi de proposer un modèle de carte qui puisse être utilisée en tant que module, de définir l'interface entre les modules afin de créer une architecture et de comprendre les comportements de calcul qui émergent de l'association des modules.


% Le chapitre 1 présente une zoologie des architectures de cartes de Kohonen existant dans la littérature.



Les travaux présentés dans cette thèse ont fait l'objet de deux présentations en conférence~:
\begin{itemize}
    \item Consensus driven ...., ICONIP 2020
    \item Input prediction in SOMs, ISCMI 2021
\end{itemize}

% A placer : 

% Apprentissage supervisé / non supervisé définition.
% Mémoire associative/traitement de séquences 


% \begin{itemize}
%     \item Apprentissage non supervisé, quantification vectorielle : définition + apprentissage développemental ? Pq c'est cool de continuer a etudier les SOM ?
%     \item Modularité et modularité dans les programmes informatiques,définition
%     \item Systèmes dynamiques complexes~: Biologique first puis exemple automates cellulaires qui sont une machine de turing, réseaux de Hopfields, machine de bolztmann: comporements de calcul comme émergeance
%     \item Systèmes d'apprentissage modulaire : des systèmes complexe.
% \end{itemize}

% \cite{Oudeyer2010OnTI} : biologie liée dans les deux sens à l'aspect computationnel.


% But : montrer comment un mécanisme de recherche de consensus entre cartes de Kohonen permet de construire des architectures apprenant des relations multimodales.

% \section{Comment construire un système modulaire d'apprentissage non supervisé}

% La modularité présente des aspects avantageux voire optimaux. 
% Une structure de réseau en petit monde est  utilisée dans des systèmes d'information, comme des bases de données, comme une structure optimisant la vitesse des échanges d'information dans le système. Or, ces structures sont également observées dans de nombreux domaines expérimentaux : biologie (exemple ?) et dans des systèmes sociaux tels que les arbres de connaissances entre individus. 

% D'un point de vue informatique, la modularité possède également de nombreuses définitions.
% On peut définir la modularité comme la décomposition d'un système en sous-systèmes plus petits et fonctionnant indépendamment. Ces systèmes interagissent via une interface bien définie.
% Plusieurs façons de construire ces sous-systèmes.
% Modules ayant des fonctions et structures différentes prédéfinie, exemple en programmation.
% Modules de même structure se différenciant au cours d'une interaction avec l'environnement (apprentissage). Dans ce cas de figure ces composants sont conçus pour être interchangeables et réutilisables.
% La modularité permet alors à un système d'être robuste à un dysfonctionnement d'un composant, sa mise à l'échelle et une faculté d'adaptation.

% Un système modulaire n'est pas forcément un système complexe.
% Cependant, lorsque l'interaction des modules est non linéaire, le système mène à l'émergence de nouveaux comportements.

% \section*{Modularité et Conception d'architectures d'apprentissage}

% Architecture d'apprentissage non supervisé = apprentissage de représentation. Passe par l'encodage de features de l'espace d'entrées sur des systèmes de calcul. 

% Les systèmes d'apprentissage existants combinent la notion de modularité et d'émergence pour former des comportements d'apprentissage plus complexes.
% Les réseaux de deep learning, se sont éloignés du modèle biologique du neurone pour ajouter des règles de calculs plus informatiques comme la backpropagation. Cependant l'approche modulaire est resté une constante dans le développement des réseaux utilisés actuellements comme le modèle teacher student et les réseaux de neurones adversarials qui combinent des réseaux performant chacun une sous-tâche par rapport à l'autre.

% La conception de système modulaire d'apprentissage peut passer par deux approches. D'un côté, une approche “finale” dans laquelle la finalité, l'application du système est connue. La conception du système passe alors par la décomposition de l'objectif en sous-systèmes et sous-tâches de manière à définir des modules particuliers.
% L'approche inverse serait l'approche constructive, dans laquelle nous disposons de modules permettant des comportements simples de calculs et les associons entre eux pour former un système modulaire. Il est difficile de connaître à l'avance le comportement final de ce type de système et son étude passe donc par la simulation.
% Cette approche, si elle n'est pas la plus directe en termes de résultats applicatifs, a l'avantage d'ouvrir la porte à des comportements d'apprentissages qui peuvent être inattendus. 

% Exemples d'archi modulaires d'apprentissage ?

% \subsubsection*{Mémoire associative et modularité}


% \subsection*{ Les cartes auto-organisatrices comme choix de modules : problématique de la thèse }