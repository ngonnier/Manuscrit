\chapter*{Conclusion}

%\section*{Discussion}

% 1 - Rappeler en trois phrases les contributions principales de la thèse.    
%     Proposition d'un modèle d'architecture modulaire de cartes auto-organisatrices, CxSOM
%     Elaboration de méthodes et d'outils de représentation adaptés à la mise en évidence des comportements d'apprentissage.
%     Identification des mécanismes principaux, en vue du développement d'une architecture comportant de plus nombreuses cartes.

La thèse que nous avons présentée propose un modèle d'architecture modulaire non-hiérarchique d'apprentissage, en utilisant des cartes de Kohonen en tant que modules~: CxSOM, pour \emph{consensus-driven multi-som}.
Par architecture modulaire, nous entendons qu'un module d'apprentissage n'a accès qu'à une interface bien définie comme connexion aux autres modules. Nous voulions également pouvoir intégrer des connexions rétroactives entre les cartes afin d'apporter l'aspect non-hiérarchique, inspiré des rétroactions existant dans le cortex cérébral.
Le modèle d'architecture a été élaboré dans une démarche synthétique~: à partir des architectures existantes dans la littérature et des modèles étudiés précédemment dans l'équipe, nous avons proposé le modèle d'architecture de cartes auto-organisatrices CxSOM, puis avons étudié son comportement sur des blocs de base, en vue d'un développement futur.
Dans cette optique, le chapitre~\ref{chap:architectures} se présente comme une revue des modèles existants d'architectures hiérarchiques et non-hiérarchiques construites à base de cartes auto-organisatrices. Nous avons unifié les notations de ces modèles, issus de différents domaines de l'informatique, et précisé leurs appellations hiérarchiques et non-hiérarchiques. Cette revue nous a permis d'identifier les structures communes à ces architectures. Nous avons en particulier noté que les différences principales se situent au niveau du choix de l'interface, effectuée en transmettant le poids du BMU, position du BMU ou un ensemble d'activations. La deuxième différence principale est dans la gestion des rétroactions pour les architectures non-hiérarchiques. Dans la plupart des modèles, les rétroactions sont en fait découplées lors de l'apprentissage puis couplées seulement dans un sens lors de phases de prédiction, ce qui s'éloigne de l'idée d'une architecture autonome.


La première contribution de cette thèse est l'élaboration d'un nouveau modèle d'architecture, CxSOM.
Dans CxSOM, nous avons choisi d'utiliser la position du BMU comme seule information transmise entre les cartes. Cette position est une valeur 1D ou 2D. Cette position est légère dans les calculs, utilise pleinement les propriétés de représentation topographiquement ordonnée des cartes de Kohonen. Il s'agit de plus d'une valeur homogène entre tous les modules de l'architecture, quelles que soient les entrées externes de ces modules et leur architecture interne, comme leur nombre de couches.
La position du BMU a été utilisée en tant qu'information transmise entre cartes dans de nombreux travaux portant sur les architectures hiérarchiques de cartes comme HSOM \parencite{lampinen_clustering_1992}, cherchant à trouver des motifs dans les données à un niveau plus abstrait ou dans les cartes récurrentes comme SOMSD \parencite{hagenbuchner_self-organizing_2003}. 
Ces travaux ont montré que la position du BMU est une information suffisante à transmettre entre cartes pour qu'elles puissent effectuer un apprentissage de façon conjointe.
Aucun travail à notre connaissance n'avait utilisé uniquement la position du BMU comme interface pour construire des architectures de cartes comportant des rétroactions, ce qui a motivé les travaux proposés dans cette thèse.
De plus, nous avons introduit un mécanisme original d'interface entre carte par une recherche de consensus entre les cartes d'une architecture, la relaxation.
La recherche du BMU correspond alors une recherche d'une position maximisant l'activité de chacune des cartes, qui apporte un comportement dynamique à l'architecture de cartes.
Le modèle CxSOM est détaillé au chapitre \ref{chap:modele}, puis le mécanisme de relaxation a été observé de façon détaillée au chapitre \ref{chap:relaxation} afin de d'appuyer les comportements validant la relaxation en tant que recherche de BMU. Ensuite, les chapitres~\ref{chap:repr}, \ref{chap:analyse} et \ref{chap:analyse2D} présentent les comportements d'apprentissage et les paramètres importants à considérer dans des architectures.

La seconde contribution de la thèse est l'élaboration d'une méthode d'étude de l'architecture et d'analyse de son apprentissage. 
Il nous a tout d'abord fallu déterminer un domaine d'étude des mécanismes de l'architecture générique que nous avons proposée. Nous avons choisi de nous intéresser à l'apprentissage associatif de données multimodales, l'aspect multisensoriel étant une motivation pour créer des architectures non-hiérarchiques. Chaque carte de l'architecture prend alors une entrée externe.
Le but de l'apprentissage associatif pour chaque module d'apprendre une représentation de leur entrée externe, tout en apprenant les relations entre les entrées au sein de l'architecture.

Dans cette optique d'évaluation d'un apprentissage associatif, nous avons proposé des représentations adaptées, afin de mettre en évidence et de quantifier comment CxSOM encode les relations entre entrées au sein de l'architecture.
Tout d'abord, nous avons présenté une méthode d'analyse de l'architecture de cartes en modélisant les entrées multimodales et les réponses des cartes comme des variables aléatoires obtenues lors de phases de test. Cette méthode est présentée au chapitre \ref{chap:repr}.
Nous avons modélisé les relations entre les entrées sous forme d'une variable latente, $U$, paramétrant le modèle sans perte d'information~: $U$ est en bijection avec l'ensemble des entrées externes $(\inpx\m{1}, \cdots, \inpx\m{n})$. 
Le but d'utiliser $U$ est d'avoir une variable représentant tout le modèle d'entrée.
La mise en évidence de l'apprentissage d'une relation entre les entrées par l'architecture revient alors à chercher comment l'architecture a encodé $U$ au sein des cartes.
Ensuite, nous avons souligné l'importance de visualiser une organisation dans les réponses des cartes, en particulier leurs BMU, et non seulement dans les poids des cartes comme la méthode classique d'évaluation des SOM. La mise en évidence d'un apprentissage passe par la représentation des dépendances entre les positions des BMU et les entrées $(\inpx\m{1}, \cdots, \inpx\m{n})$ et $U$.

\`A partir de cette méthode d'analyse, nous avons étudié le comportement d'architectures élémentaires de deux et trois cartes en une dimension, apprenant sur des entrées en une dimension. Cette étude contribue à l'objectif de conception de l'architecture, et vise à identifier ls mécanismes d'apprentissage émergeant de l'architecture de cartes, la dépendance de l'architecture aux paramètres d'apprentissage et les limites et perspectives éventuelles du modèle d'architecture.
Ces expériences sont présentées au chapitre \ref{chap:analyse} et \ref{chap:analyse2D}.
Nous avons dégagé quelques comportements qui nous apparaissent comme inhérents aux règles de calculs définies dans CxSOM.
Tout d'abord, pour faire émerger un apprentissage du modèle d'entrée et non seulement de l'entrée externe, nous voulons prendre un grand rayon de voisinage externe $r_e$ face au rayon contextuel $r_c$. 
Cette différence d'échelle entre paramètres induit une organisation subordonnée des poids contextuels face aux poids externes lors de l'apprentissage, conduisant les cartes à s'organiser selon deux échelles d'indices. Une carte s'organise ainsi globalement selon la valeur de ses entrées externes, et au sein de cette quantification localement la position des BMUs selon la valeur du modèle d'entrées $U$. Cette organisation se traduit par une disposition pseudo-périodique des poids contextuels de la carte.
Cette organisation à deux échelles intervient lorsqu'une carte doit différencier une même valeur de son entrée externe qui correspond à plusieurs points différents du modèle d'entrées. 
Cette organisation à deux échelles réalise un compromis entre l'encodage de $U$ et la qualité de la quantification vectorielle sur $\inpx$, dont la qualité est réduite par rapport à une carte classique. Dans le cas ou le modèle d'entrée ne nécessite pas cette séparation, une carte se comporte comme une carte classique.
Cette encodage de $U$ dans chaque carte est alors caractérisé par le fait que $U$ est directement une fonction de la position du BMU $\bmu$.
    
Grâce à l'encodage de $U$ dans chaque carte, encodant l'information de façon redondante, une architecture CxSOM est capable de générer une prédiction dans une des cartes de l'architecture à laquelle on n'a pas présenté d'entrée externe lors du test. Grâce aux rétroactions, une carte acquiert une capacité de prise de décision sans avoir besoin d'un algorithme supplémentaire analysant la sortie des cartes. La prédiction s'effectue localement, au niveau d'une carte~: les autres cartes de l'architecture n'ont pas besoin de savoir si une carte prend ou non son entrée externe. Enfin, grâce aux rétroactions, la prédiction est possible pour n'importe quel carte de l'architecture. Cette capacité n'est pas permise par des architectures hiérarchiques ou des cartes classiques.
   
Enfin, nous avons mis en évidence au chapitre~\ref{chap:analyse2D} que le comportement généré par les cartes en une dimension s'étend bien aux cartes en deux dimensions, qui sont généralement utilisées en pratique. Ce comportement est prometteur pour la mise en pratique des architectures de cartes sur des données de plus grande dimension.

Au chapitre~\ref{chap:indicateur}, nous avons enfin étudié des mesures statistiques permettant de quantifier l'apprentissage des relations entre entrées par une architecture de cartes.
Ces quantités s'appuient sur la modélisation des entrées en tant que variables aléatoires et sur la variable latente représentant le modèle, $U$.
Nous avons exploré dans cette thèse deux coefficients quantifiant la propriété que $U$ est une fonction du BMU dans chaque carte. Le premier est coefficient d'incertitude, s'appuyant sur l'information mutuelle. Nous avons conclu qu'il n'est pas l'indicateur le plus adapté à l'évaluation de la relation fonctionnelle entre $U$ et $\bmu$.
Le second indicateur est le ratio de corrélation. Cet indicateur est bien adapté à l'évaluation d'une relation fonctionnelle, en notant toutefois que sa valeur devra être utilisée en comparaison à des valeurs d'entrées, et non prise de manière absolue. Cet indicateur numérique permettra de comparer des expériences entre elles et d'optimiser automatiquement les paramètres d'apprentissage de l'architecture.
Nous avons noté qu'il peut être intéressant d'envisager des indicateurs permettant de transcrire l'organisation des cartes d'une architecture, ce qui n'est pas effectué par le ratio de corrélation.
Par ailleurs, la propriété que $U$ est une fonction du BMU est certes observée pour deux et trois cartes, mais n'est pas souhaitable pour des grandes architectures et des entrées de plus grande dimension.
On souhaiterait plutôt que l'information sur $U$ soit distribuée entre les différentes cartes de l'architecture, tout en gardant de la redondance pour permettre la capacité de prédiction d'entrée.
Dans ce cas, les deux indicateurs proposés dans ce chapitre ne transcriront pas cette propriété et nous suggérons aux travaux futurs de s'intéresser à l'information mutuelle entre les caractéristiques des cartes, notamment les BMUs, et les entrées. Nous avons par exemple mis en évidence par le calcul de l'information mutuelle entre $U$ et $\bmu$ que chaque carte perd globalement de l'information sur le modèle $U$ par rapport à une carte apprenant seulement l'entrée externe $\inpx$.
Cette perte d'information est due à la perte de précision sur la quantification vectorielle de l'entrée externe~: en effet, $\inpx$ porte de l'information sur $U$, mais elle cache un gain d'information sur le modèle $U$.
On voudra par exemple pouvoir mesurer seulement le gain d'information sur $U$ dans une carte, ou dans toute l'architecture.

Le modèle que nous avons proposé semble bien se généraliser sur des entrées de dimensions supérieures et sur des architectures comprenant plus de cartes.
Nous pouvons envisager certaines limitations générales du modèle, qui seront des pistes d'étude possible pour une amélioration ou une application de l'architecture.
Tout d'abord, le fait d'encoder $U$ dans chaque carte apporte de la redondance et permet la prédiction, mais il n'est pas souhaitable que $U$ soit complètement encodé dans chaque carte dans un cas général d'architecture.
Cela induirait que la valeur encodée ne peut pas dépasser une ou deux dimensions correspondant à la dimension des cartes, limite qui ne pourrait être améliorée par l'augmentation du nombre de cartes, rendant l'architecture quelque peu inutile. On voudrait plutôt que les différentes cartes apprennent collectivement une représentation distribuée de $U$, ce qui laisse la possibilité au modèle d'améliorer l'apprentissage des relations entre entrées en ajoutant des cartes à l'architecture. Cette limitation potentielle n'a pas pu être étudiée dans des architectures de deux et trois cartes, mais il s'agira d'un point à vérifier et observer lors de l'étude de plus grandes architectures.
% On observe par ailleurs qu'une valeur de $U$ de grande dimension est difficile à encoder par les cartes du fait d'une grande rigidité entre les couches de poids. Les poids contextuels se moyennent lorsqu'ils cherchent à apprendre une valeur de $U$ en dimension 4, par exemple dans l'hypercube 4D au chapitre \ref{chap:analyse2D}. 
% Ce comportement peut poser problème si les relations que l'architecture cherche à encoder se répartissent sur de nombreuses dimensions, c'est-à-dire quand $U$ est de grande dimension. Cette observation rejoint le fait qu'on veut que l'apprentissage de $U$ soit distribué au sein de l'architecture.
Ensuite, la double échelle de quantification vectorielle induit beaucoup de n\oe{}uds morts dans la carte, donc une perte d'unité d'apprentissage. Les n\oe{}uds morts sont nécessaires pour créer la double échelle de quantification vectorielle, qui permet l'apprentissage du modèle et la prédiction. En effet, une carte de Kohonen par construction garde une continuité entre les valeurs des prototypes.
Pour garder cet aspect discontinu mais enlever les n\oe{}uds morts, on pourrait par exemple ajouter des poids dans les arêtes des cartes, qui permettraient de simuler les n\oe{} morts dans le calcul et de faire en sorte que les n\oe{}uds de la carte soient tous des BMUs.
    

En conclusion, le modèle CxSOM proposé dans cette thèse apporte un nouveau paradigme de construction d'architecture modulaire, utilisant pleinement la représentation topographiquement ordonnée d'une carte auto-organisatrice.
Dans ce modèle, chaque carte encode une représentation de son entrée externe et les relations entre les entrées sont également encodées dans chaque carte de l'architecture. 
Nous avons mis en lumière que cet apprentissage et les rétroactions permettent un comportement original pour des cartes auto-organisatrices~: une carte de l'architecture est capable de générer une valeur à partir de ses connexions contextuelles. Cette valeur correspond à l'entrée qui n'a pas été présentée, il s'agit donc d'une prédiction.
Lors de cette thèse, nous avons défini une méthode d'analyse de la réponse des cartes en vue d'une étude de plus grandes architectures, extrait les caractéristiques de l'organisation marquant l'apprentissage, et identifié des points à surveiller lors d'un passage à des plus grandes architectures.

\section*{Perspectives}

Les perspectives à court terme de ces travaux sont de continuer le développement du modèle en s'intéressant aux connexions au sein d'une architecture comportant plus de deux et trois cartes.
Le nombre de connexions possible au sein d'une architecture comportant un nombre fixé de cartes croît exponentiellement avec le nombre de cartes et chaque configuration de connexions peut complètement modifier la façon dont se comporte l'architecture. Par ailleurs, certaines cartes peuvent ou non prendre des entrées externes, ajoutant un grand nombre de configurations possibles à explorer.
Grâce à nos travaux, nous avons pu identifier les paramètres importants à utiliser dans l'architecture et les caractéristiques des cartes pertinentes à considérer pour étudier les comportements d'apprentissage de relations multimodales. L'étude de plus grandes architectures devra se faire d'un point de vue plus global, en s'appuyant sur le comportement général de l'architecture et non seulement d'un point de vue d'une carte. 
Non des moindres, les travaux d'analyse du comportement des architectures dont nous présentons ici les résultats ont été accompagnés d'un travail d'élaboration, en collaboration avec Hervé, d'une librairie C++ et python permettant la gestion des nombreuses variables d'état qui composent une architecture de cartes. Cette librairie est maintenue par Hervé Frezza-Buet \footnote{\href{https://github.com/HerveFrezza-Buet/cxsom}}. La mise en relation de nombreuses cartes est en effet 


Il reste également à définir des cas d'études sur lesquels appliquer ces architectures à grande échelle.
L'aspect modulaire de ces architectures pourrait par exemple nous faire envisager des modules d'interaction avec l'environnement, qui traitent les entrées sensorielles et des modules d'apprentissage, en s'inspirant des structures fonctionnelles observées en biologie \parencite{Ellefsen2015NeuralMH}. 
Des modélisations récentes du cortex sous forme de réseaux mettent l'accent sur son aspect modulaire multi-échelle \parencite{betzel_multi-scale_2017}~: le cortex semble s'organiser en une architecture dont les modules sont eux-mêmes des architectures modulaires, loin des trois modules que nous avons étudiés dans cette thèse , et motive donc la construction d'architectures de bien plus grande ampleur.

Un second objectif du développement d'architectures multi-cartes est également l'intégration de connexions récurrentes entre cartes, afin de traiter des données séquentielles conjointement avec leur aspect spatial. 
L'utilisation de la position du BMU comme interface a été utilisée au sein de modèles de cartes récurrentes telles que SOMSD~; ce modèle ainsi que son adaptation sur deux cartes ont fait l'objet d'études précédentes dans notre équipe \parencite{baheux_towards_2014, fix20}. 
Dans ces modèles de cartes récurrentes, la carte prend en entrée externe un élément d'une séquence d'entrée et comme entrée contextuelle la position du BMU obtenu lors de l'itération précédente.
Les propriétés d'organisation observées sur ce type de cartes récurrentes rejoignent celles observée dans l'architecture CxSOM~: une carte distingue son BMU en fonction de l'entrée externe, mais également en fonction de sa place dans la séquence d'entrée.
Une perspective d'étude sera ainsi d'associer des connexions temporelles et des connexions multimodales au sein d'une architecture de cartes afin de traiter des données séquentielles.
Un inconvénient des cartes récurrentes simples est le fait qu'elles oublient une séquence, une fois que cette dernière n'est plus présentée. 
Une architecture de cartes pourrait par exemple apporter des modules de mémoire supplémentaire pour l'apprentissage d'un ensemble de séquences et non d'une seule.
Une direction d'application d'architectures de cartes peut enfin être la construction d'un système d'apprentissage \og sur le long terme \fg{}, apprenant au cours du temps tout en étant capable de générer des prises de décision dans le système.



% Enfin, l'architecture que nous avons proposée s'appuie uniquement sur des cartes auto-organisatrices. Nous pouvons envisager de coupler les mécanismes induits par l'architecture de cartes à d'autres mécanismes d'apprentissage non-supervisés, comme de l'apprentissage par renforcement ou des auto-encodeurs.

% % Dernière phrase choc ?

% Finalement, l'utilisation d'une information aussi simple que la position du BMU comme interface entre les cartes d'une architecture a permis d'observer des comportements assez complexes. 

